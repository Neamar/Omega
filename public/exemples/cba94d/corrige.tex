\documentclass[a4paper,10pt]{article}
\usepackage[utf8x]{inputenc}
\usepackage[T1]{fontenc}
\usepackage[francais]{babel}
\usepackage{lscape,latexsym,graphicx,cancel,amsmath,color,textcomp}
\graphicspath{{Ressources/}}

\newcommand{\eDevoir}{\rotatebox[origin=c]{15}{e}Devoir}
%TODO : Faut-il protéger le titre d'inclusion spéciales ? Accolade fermante probablement déjà.
\title{Équations}
\author{\eDevoir}

\begin{document}
%TODO : quelle balise modifier ?
%TODO : interdire l'inclusion de ressources externes
\maketitle
\section{Équation n°1}

$$ -5(2+4x)=-6(2-3x)$$
Je multiplie les deux membres par $-1$, ce qui me permet d'éliminer les deux $-$.
$$ 5(2+4x)=6(2-3x)$$
Je développe
$$ 10+20x=12-18x $$
Je passe tous les $x$ du même côté, c'est à dire que j'ajoute $18x$ à chaque membre de l'équation
$$ 10+20x+18x=12-\cancel{18x}+\cancel{18x}$$
$$ 10+38x=12$$
Je passe $10$ de l'autre côté en retirant $10$ à chaque membre.
$$ \cancel{10}+38x-\cancel{10}=12-10$$
$$ 38x=2$$
Je divise par $38$ qui n'est pas nul. Cela peut sembler évident mais il faut avoir le réflexe avant chaque division de vérifier qu'on ne divise pas par $0$ !
$$ \boxed{x=\frac{1}{19}}$$

\section{Équation n°2}
$$9-3(5+2x)=4+8(1-3x)$$
Je développe
$$9-15-6x=4+8-24x$$
$$-6-6x=12-24x$$
Les $x$ passent à gauche, c'est à dire que j'ajoute $24x$ des deux côtés.
$$-6-6x+24x=12-\cancel{24x}+\cancel{24x}$$
$$-6+18x=12$$
Le $-6$ passe à droite en ajoutant $6$ aux deux membres.
$$-\cancel{6}+18x+\cancel{6}=12+6$$
$$18x=18$$
Reste à diviser par $18$ qui n'est pas nul.
$$\boxed{x=1}$$

\section{Équation n°3}
$$1-2(3x-4)=5x-2(3x-5)$$
Je développe.
$$1-6x+8=5x-6x+10$$
Je simplifie les $-6x$
$$1+8=5x+10$$
$$9=5x+10$$
Je change le $10$ de côté, c'est à dire je soustrait $10$ à chaque membre.
$$9-10=5x+\cancel{10}-\cancel{10}$$
$$-1=5x$$
Reste à diviser par $5$ qui n'est pas nul.
$$\boxed{x=-\frac{1}{5}}$$

\section{Équation n°4}
$$-x-6(-5-3x)=2(-2x+9)-5$$
Je développe
$$-x+30+18x=-4x+18-5$$
$$17x+30=-4x+13$$
Je change le $4x$ de côté en ajoutant $4x$ aux deux membres
$$17x+30+4x=-\cancel{4x}+13+\cancel{4x}$$
$$21x+30=13$$
Je change le $30$ de côté en retirant $30$ aux deux membres.
$$21x+\cancel{30}-\cancel{30}=13-30$$
$$21x=-17$$
Il reste à diviser par $21$ (qui n'est pas nul).
$$\boxed{x=-\frac{17}{21}}$$

\end{document}
