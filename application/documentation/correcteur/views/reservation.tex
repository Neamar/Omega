Depuis le \doc[correcteurs/marche]{marché aux exercices}, vous avez trouvé un exercice qui vous convenait et avez cliqué sur le lien “C’est pour moi !”.
Sur la nouvelle page qui s’affiche, indiquez l’offre que vous faites en prenant en compte le délai laissé par l’élève, la difficulté de l’exercice, le type de la demande... consultez la section suivante pour plus de détails.
Indiquez votre date d’expiration d’offre : au delà de cette limite, votre offre sera automatiquement annulée si l’élève n’a toujours pas répondu. Soyez larges tout en gardant à l’esprit que si l’élève accepte dix minutes avant votre délai, vous devez encore être capable de réaliser l’intégralité de l’exercice ! N’oubliez pas qu’un simple retard entraîne \doc[correcteurs/bloque]{le blocage de votre compte}...

Vous pouvez aussi indiquer un message qui sera envoyé à l’élève. Cette section est typiquement là pour vous permettre d’indiquer à l’élève que vous accepter de faire l’ensemble du travail sauf une ou deux questions qui vous semble extrêmement complexes et que vous ne vous sentez pas capable de réaliser correctement. Cette possibilité est à utiliser avec parcimonie dans la mesure où une demande élève n’est pas conditionnelle et se doit d’être traitée dans sa globalité.

\subsection{Combien puis-je demander ?}
Il n’y a pas de règle ou de réponse absolue à cette question. Le tout, c’est de faire une offre qui ne soit pas surévaluée par rapport au travail à accomplir - faute de quoi l’élève la refusera (voir \doc[correcteur/refus]{En cas de refus]) - et, dans votre intérêt, évitez de sous-évaluer votre travail.
Dans la pratique, c’est que si vous évaluez le temps de correction à $x$ heures de travail et que vous souhaitez être rémunéré $y$€/heure, alors vous demanderez $%EQUIVALENCE_POINT% \times x \times y$ points. Pour un exercice qui vous prendrait une heure par exemple, en souhaitant être rémunéré 15€ de l’heure, vous pourriez demander $15 \times %EQUIVALENCE_POINT%$ points.