La raison est extrêmement simple. Imaginez que vous soyez en train d’effectuer la correction d’un exercice et, arrivé à la dernière question, vous vous rendez compte que vous n’arrivez pas à y répondre. Ce cas peut sembler surprenant mais il n’est pas rare de voir, en fin de devoir maison par exemple, une question “bonus” plus complexe que la moyenne - et vous pourriez ne pas vous être rendu compte de sa difficulté lorsque vous avez fait votre offre à l’élève. Si vous envoyez une correction partielle, l’élève ne sera pas content et il ne fait aucun doute qu’il engagera une procédure de contestation. Une telle procédure est lourde de conséquence pour le correcteur puisque celui-ci peut-être bloqué (voir \doc[correcteur/bloque]{les conditions du blocage}) et prend le risque de ne plus jamais pouvoir retravailler pour le site. Afin d’éviter qu’une telle situation ne se produise, nous laissons aux correcteurs la possibilité d’envoyer gratuitement leur travail si celui-ci devait leur sembler imparfait ou incomplet.
Disons-le ces cas sont extrêmement improbables et doivent rester le plus sporadiques possible. Notre intérêt est que l’élève soit content et le correcteur payé.