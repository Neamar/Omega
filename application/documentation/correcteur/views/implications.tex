Bien entendu. Sur \eDevoir, vous n'êtes contraint à rien du tout, si ce n'est d'accepter les \doc[cgu]{Conditions Générales d'Utilisation} du site. En réalité, aucune offre ne vous est faite, c'est vous qui en faites aux élèves en fonction de leurs demandes. Vous êtes libre de ne faire qu'une offre par an si cela vous chante !

Par contre, une fois une offre acceptée par un élève, vous vous engagez auprès de nous à réaliser ce travail dans les temps et correctement (voir \doc[correcteur/retard]{que se passe-t-il en cas de retard} ?). Vous vous engagez également à fournir un travail de qualité en respectant la charte du site (voir \doc[correcteur/qualite]{que se passe-t-il si mon travail n'est pas de qualité ?}). Enfin, vous vous engagez à suivre l'élève après l'envoi de la correction, dans la FAQ privée qui s'ouvre entre vous deux (voir \doc[correcteur/faq]{la faq}).