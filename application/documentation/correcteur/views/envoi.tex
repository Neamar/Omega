Depuis \l[/correcteur/]{votre page d'accueil}, recherchez l'exercice auquel vous souhaitez répondre puis cliquez sur le lien "Je m'y mets". Vous arrivez dès lors sur la page de rédaction.

Rentrez votre corrigé au format $\LaTeX$ (\doc[correcteur/tex]{pourquoi} ? \doc[correcteur/aide_tex]{comment} ?).
Vous pouvez vérifier le rendu de votre texte à tout moment en cliquant sur le bouton "Prévisualiser".
Si nécessaire, vous pouvez envoyer des fichiers image (format acceptés : PNG, EPS, SVG, JPG). Ceux-ci deviennent alors disponibles pour inclusion dans votre rendu : vous pourrez y accéder dans le dossier \texttt{assets/}.

Par exemple, si vous envoyez une image nommée Image\.png, vous pourrez l'inclure dans votre PDF avec cette ligne de commande :
\begin{verbatim}
\includegraphics{assets/Image.png}
\end{verbatim}
Vous pouvez aussi l'inclure plus facilement simplement en cliquant sur le nom du fichier en dessous de l'interface $\LaTeX$.

Si vous envoyez un fichier portant le même nom qu'un ancien fichier déjà envoyé, l'ancien est écrasé sans confirmation.

Une case à cocher vous permet d'envoyer votre travail gratuitement (\doc[correcteur/envoi_gratuit]{consultez la page sur l'envoi gratuit pour des détails sur la procédure}).

Une fois satisfait du résultat, vous pouvez cliquer sur le bouton "Envoyer". Attention, il s'agit d'un envoi définitif : vous ne pourrez plus rien modifier après et le fichier sera directement envoyé à l'élève ! Vérifiez donc bien l'absence d'erreur (sous peine de faire l'objet d'une \doc[correcteur/contestation]{demande de remboursement} de l'élève) et l'absence d'éléments personnels (voir la page sur \doc[correcteur/alerte]{les alertes}).

Une fois le texte envoyé, l'exercice n'est plus considéré comme réservé et ne compte donc plus dans la limite de fichiers (voir \doc[correcteur/limite_reservation]{limitation des réservations}).