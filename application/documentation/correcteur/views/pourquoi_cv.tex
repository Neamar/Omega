En fournissant le CV, nous pouvons vérifier vos compétences.
Nous ne lirons pas le document "dans le but d'une embauche", mais uniquement pour évaluer vos capacités et votre cursus. Il s'agit aussi d'une étape permettant d'éliminer les personnes trop jeunes qui n'ont pas un cursus satisfaisant (lycéens ou autres) et ne devraient donc pas être admis en tant que correcteurs sur ce site.
En conséquence, nous pouvons tolérer un CV non mis à jour (par exemple suite à votre dernière embauche) du moment que le reste du cursus établit vos compétences.

Le CV fourni doit-être, faut-il le préciser, au même nom que la personne possédant l'auto-entreprise. Des vérifications sont faites afin d'éviter tout abus.

Nous rappelons que la loi rend passible d'amende et/ou d'emprisonnement quiconque se rend coupable de fraudes ou de fausses déclarations (L4441-1 du code Pénal). Nous vous engageons donc, lors de l'inscription, à nous fournir des documents de qualité... et véridiques !