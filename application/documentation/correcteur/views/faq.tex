Dans un souci de qualité et de suivi des élèves, \b{eDevoir} met en place un système de FAQ entre les élèves et les correcteurs. Cette FAQ n'est pas seulement une annexe du système, elle en est une de ses composantes principales qui ne doit pas être négligée.

En acceptant de réaliser l'exercice, vous vous engagez aussi à répondre de façon active et pertinente aux questionnements de l'élève via cette FAQ. À chaque fois qu'une nouvelle question y sera postée par l'élève, un mail vous sera envoyé. Nous attendons de vous que vous preniez du temps pour répondre à ces questions, dans la limite du raisonnable bien sûr.
Si vous ne répondez pas aux questions ou si vous y répondez mal, l'élève peut vous signaler, entraînant dès lors un examen par nos soins de votre cas. Si à l'inverse l'élève exagère en vous posant par exemple des questions déconnectées du devoir pour lequel vous avez été payé, il vous est possible de le signaler par le bouton prévu à cette effet dans la page.

N'oubliez pas que nous sommes en France et que le kikoolol n'est pas la langue officielle. Tous vos textes se doivent d'être correctement rédigés, et les formules mathématiques mises en forme à l'aide de LaTeX (\doc[tex]{aide pour la rédaction sur la FAQ exercice}).