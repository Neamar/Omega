Tout à fait. Le souci de \eDevoir est de fournir un rendu professionnel et de qualité aux élèves et parents qui s' y inscrivent. Par conséquent, nous attachons une grande importance aux éléments suivants :
\begin{itemize}
	\li Que le travail soit rendu dans les temps ;
	\li qu'il soit de qualité et rédigé proprement ;
	\li que le correcteur assume le suivi de ce devoir (et donc de l'élève) via la FAQ spécialement ouverte à cet effet entre les différents acteurs de cet échange.
\end{itemize}

Par conséquent, un élève peut à tout moment lancer une procédure de réclamation à l'encontre d'un correcteur, procédure relative tant au corrigé qu'au suivi du travail dans la FAQ.

Si une procédure est lancée, une équipe compétente analysera cette contestation et jugera de son bien-fondé. Elle pourra ainsi demander le remboursement total ou partiel des frais engagés par le client. Ce manque à gagner incombera bien entendu au correcteur. Dans les faits, nous achèterons son corrigé moins cher que le tarif souhaité (ou nous déciderons de ne pas l'acheter du tout dans le cas où une demande de remboursement totale serait acceptée). Bien entendu cela n'est absolument pas dans notre intérêt et nous souhaitons au maximum garder nos correcteurs. Un incident peut arriver tout à fait exceptionnellement, mais si les erreurs venaient à se répéter, nous pourrions alors rompre définitivement toute collaboration avec le correcteur concerné.