Si vous ne rendez pas le corrigé dans les temps, vous ne serez tout simplement pas payé. De plus, dans un souci de satisfaction client (donc élève), \eDevoir remboursera l'élève à hauteur de __POURCENTAGE_RETARD__\% (voir \doc[eleve/retard]{que se passe-t-il en cas de retard ?}). Il n'est donc tout simplement pas envisageable que le travail ne soit pas rendu dans les temps. Prévoyez toujours un marge respectable en gardant à l'esprit qu'une coupure internet est envisageable, même dans notre beau pays civilisé !

Un retard sera donc assez mal vu de la part d'eDevoir, et vous pourriez voir votre compte bloqué définitivement (voir \doc[correcteur/bloque]{compte bloqué}). En effet, non seulement le site perd de l'argent, mais en plus l'élève n'est pas satisfait. Nous vous encourageons donc, même en cas de retard, à quand même envoyer le travail réalisé, même si celui-ci est hors délai et que vous ne serez par conséquent pas rémunéré pour celui-ci. Ceci est un gage de bonne foi pour nous.