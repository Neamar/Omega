Le paiement d'un exercice suit la procédure suivante :

\begin{itemize}
	\li Vous envoyez votre correction à l'élève ;
	\li jusqu'à __DELAI_FAQ__ jours après la date d'expiration élève, l'exercice n'est toujours pas clos et peut faire l'objet d'une contestation (voir \doc[correcteur/contestation]{procédure de contestation}). Durant cette période, vous vous devez d'assurer le suivi de l'élève via \doc[correcteur/faq]{la FAQ} ouverte entre lui et vous ;
	\li si __DELAI_FAQ__ jours après la date d'expiration élève la correction n'a fait l'objet d'aucune procédure de réclamation, vous êtes payé et l'argent arrive dans votre compte, \l[correcteur/points/retrait]{prêt à être récupéré}. Sinon, la procédure de réclamation peut encore retarder le paiement d'une semaine.
\end{itemize}

Si vous n'êtes pas encore payé, c'est que toutes ces étapes ne sont pas encore passées ou que vous avez fait l'objet d'une contestation approuvée par les autorités compétentes à hauteur de 100\% ou plus. Soyons clairs, ce cas est extrêmement rare et ne peut que résulter d'une grave erreur de la part du correcteur ou d'une volonté délibérée de sa part de saboter le travail. Notre but principal est de prendre soin de correcteur et que ceux-ci soient heureux de travailler avec \eDevoir. Il n'est donc nullement dans nos intentions de ne pas vous rémunérer.

En conclusion, si tout se passe bien et qu'aucune réclamation élève n'est déclenchée, votre compte \eDevoir sera crédité au plus tard 14 jours après la fin de la deadline élève.