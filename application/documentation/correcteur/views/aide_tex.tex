Petit guide du LaTeX pour néophytes !

\section{Structurer son document}
Anfin de structurer votre devoir, il vous est recommandé d'utiliser les commandes suivantes :
\begin{verbatim}
\section{}
\subsection{}
\subsubsection{}
\paragraph{}
\subparagraph{}
\end{verbatim}
Les titres de parties ainsi obtenus sont numérotés automatiquement. Si donc vous bouleversez toute l'organisation de votre texte, la numérotation se refera d'elle-même à la compilation suivante. Ainsi, il ne faut pas taper \verbatim{\section{1. MaPremièreSection}} mais \verbatim{\section{MaPremièreSection}}.

\section{Polices}
Pour définir un bloc, on procède de la manière suivante :
\begin{verbatim}
{ \commande Texte... }
\end{verbatim}
Dans cet exemple, la commande s'applique à tout le bloc.

Voici les tailles de polices standards, de la plus petite à la plus grande :
\begin{verbatim}
\tiny,
\scriptsize,
\footnotesize,
\small,
\normalsize,
\large,
\Large,
\LARGE,
\huge,
\Huge.
\end{verbatim}

Les styles les plus utiles que vous aurez à utiliser sont les suivants :
\begin{verbatim}
\bf %Gras
\it %Italique
\texttt %Machine à écrire
\rm %Roman (texte normal)
\end{verbatim}

Pour note vous pouvez aussi utiliser les commandes sans bloc telles que :
\begin{verbatim}
\textbf{Texte...} %Gras
\textit{Texte...} %Italique
% etc.
\end{verbatim}

Ainsi, les deux commandes suivantes seront équivalentes :
\begin{verbatim}
\textbf{Lorem Ipsum}
% et
{\bf Lorem Ipsum}
\end{verbatim}

Pour aller à la ligne, il sera nécessaire d'utiliser la commande \verbatim{double back-slash \\}.

\section{Les listes}
On peut faire une liste d'objets avec les environnements itemize (liste simple) ou enumerate (liste numérotée), de la façon suivante :
\begin{verbatim}
Le correcteur eDevoir est :
\begin{itemize}
	\item sérieux ;
	\item rapide ;
	\item efficace.
\end{itemize}
\end{verbatim}

Ou bien encore :
\begin{verbatim}
Les formules de trigo sont :
\begin{enumerate}
	\item $\sin (A - B) = \sin A \cos B - \cos A \sin B$ ;
	\item $\sin (A + B) = \sin A \cos B + \cos A \sin B$ ;
	\item $\cos (A - B) = \cos A \cos B + \sin A \sin B$ ;
	\item $\cos (A + B) = \cos A \cos B - \sin A \sin B$.
\end{enumerate}
\end{verbatim}

Les deux environnements ci-dessus peuvent être encapsulés pour créer des listes à plusieurs niveaux.
\begin{verbatim}
\begin{enumerate}
   \begin{enumerate}
      \item Texte
      \item Texte
      \item etc...
   \end{enumerate}
   ...
\end{enumerate}
\end{verbatim}
Dans le cas de l'enumerate, il pourra changer (suivant le style) de type de compteur (1 2 3, a b c, i ii iii ...) suivant la profondeur.

\section{Environnements modifiant la justification des paragraphes}
Les environnements s'ouvrent avec \verbatim{\begin{***}} et se referment avec \verbatim{\end{***}}, en faisant figurer le nom de l'environnement entre les accolades. Si vous utilisez plusieurs environnements emboîtés (listes imbriquées, citations centrées, etc), veillez à bien les refermer dans l'ordre de l'imbrication : dernier ouvert, premier refermé, comme les parenthèses en mathématiques.
\item \texttt{flushleft} : aligner le texte sur la marge de gauche ;
\item \texttt{flushright} : aligner le texte sur la marge de droite ;
\item \texttt{center} : centrer le texte ;
\item \texttt{quote} : faire une citation ;
\item \texttt{quotation} : faire une longue citation.

Exemple :
\begin{verbatim}
\begin{center}
   Bloc centré
\end{center}
\end{verbatim}

\section{Environnement tabular pour faire des tableaux}
\begin{verbatim}
\begin{tabular}{format colonnes}
case(1,1) & case(2,1) \\
case(2,1) & case(2,2) \\
\end{tabular}
\end{verbatim}
avec \texttt{r}, \texttt{l} ou \texttt{c} pour aligner à droite, gauche et centrer respectivement. Une barre verticale "|" permet de tracer une droite verticale dans le tableau, par exemple :
\begin{verbatim}
\begin{tabular}{|c|c|}
case(1,1) & case(2,1) \\
\hline
case(2,1) & case(2,2) \\
\end{tabular}
\end{verbatim}
mettra trois lignes verticales, autour du tableau et entre les deux colonnes. La commande \texttt{\verbatim{\hline}} permet de tracer un trait horizontal.

\section{Inclusion de figures}
La commande suivante permet d'inclure une image dans un document tex :
\begin{verbatim}
\includegraphics[options]{Fichier}
\end{verbatim}
Où l'option peut être :
\begin{verbatim}
width=\linewidth
% ou bien
width=0.5\linewidth
\end{verbatim}
Afin que la largeur de l'image soir de la largeur de la zone de texte, ou deux fois plus petite respectivement.

Il est important de préciser la chose suivante : "fichier" soit être sous la forme graph\.jpg ou graph\.png, donc avec extension, sauf pour le format pdf. Dans ce cas, le "\.pdf" ne doit pas être renseigné. Vous écrirez alors :
\begin{verbatim}
\includegraphics[width=\linewidth]{graph} % inclut le fichier graph.pdf
\end{verbatim}

Pour savoir comment ajouter des ressources à votre correction, merci de \doc[correcteur/aide_ressource]{visiter cette page}.

\section{Environnements mathématiques}
Pour une expression mathématique intégrée à un paragraphe, on utilise la commande \texttt{\verbatim{$...$}}. L'expression mathématique est donc incluse entre deux signes dollars. Pour un expression mathématique isolée - c'est-à-dire qui sera automatiquement mise à la ligne et centrée - il convient d'utiliser la commande \texttt{\verbatim{$$…$$}}. Cette formulation sera préférée pour les formules importantes par exemple.

Pour voir des exemples de constructions mathématiques, merci de \doc[tex]{visiter cette page}.