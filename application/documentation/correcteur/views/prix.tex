Il n'y a pas de règle ou de réponse absolue à cette question. Le tout, c'est de faire une offre qui ne soit pas surévaluée par rapport au travail à accomplir - faute de quoi l'élève la refusera (voir \doc[correcteur/refus]{en cas de refus}) - et, dans votre intérêt, évitez de sous-évaluer votre travail.
Dans la pratique, c'est que si vous évaluez le temps de correction à $x$ heures de travail et que vous souhaitez être rémunéré $y$€/heure, alors vous demanderez $__EQUIVALENCE_POINT__ \times x \times y$ points (__EQUIVALENCE_POINT__ points correspondant à 1€). Pour un exercice qui vous prendrait une heure par exemple, en souhaitant être rémunéré 15€ de l'heure, vous pourriez demander $15 \times __EQUIVALENCE_POINT__$ points.