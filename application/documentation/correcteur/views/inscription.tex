Rendez-vous sur la page \l[/correcteur/inscription]{d'inscription des correcteurs}.

De nombreux champs vous sont demandés pour nous assurer de votre identité :
\begin{itemize}
\li Votre prénom
\li Votre nom
\li Votre numéro de téléphone
\li Votre adresse mail
\li Votre mot de passe
\li Une version PDF de votre CV. \doc[correcteur/pourquoi_cv]{Pourquoi fournir ce document ?} Seul le PDF est accepté : cela nous permet aussi de vérifier que vous maîtrisez les bases de l'informatique et êtes capable de produire un tel document.
\li Votre numéro de SIRET. \doc[correcteur/pourquoi_autoentreprise]{Pourquoi fournir ce numéro ?} Si vous ne l'avez pas encore, vous pouvez le laisser vide pour l'instant : cependant vous n'aurez aucun moyen de recevoir un paiement tant que vous ne nous l'aurez pas communiqué. Une fois votre numéro de SIRET disponible, vous pouvez le modifier depuis vos \l[/correcteur/options/]{options}.
\li Un scan de votre carte d'identité. Vous pouvez nous fournir le même document que celui fourni lors de votre déclaration en tant qu'auto-entrepreneur.
\end{itemize}

Entrez ensuite l'image pour prouver que vous n'êtes pas un robot, puis cliquez sur Inscription. Bienvenue parmi les correcteurs !

\subsection{Pourquoi l'inscription demande-t-elle autant d'informations personnelles ?}
L'inscription en tant que correcteur est volontairement plus complexe que l'\doc[eleve/inscription]{inscription en tant qu'élève}. En effet, les correcteurs ont des contraintes et des responsabilités plus importantes. Nous ne souhaitons avoir que des personnes compétentes et motivées : c'est pourquoi nous vérifions la pertinence des informations fournies. Cette contrainte explique aussi le \doc[correcteur/validation]{délai de validation}.
Légalement, il est obligatoire d'identifier les émetteurs de facture.
Enfin, en complexifiant le formulaire, nous nous assurons de la motivation des correcteurs et évitons les abus.

Attention, pour être correcteur sur eDevoir, vous devez être une personne morale, c'est à dire posséder une entreprise individuelle ou auto-entreprise (voir \doc[correcteur/pourquoi_autoentreprise]{pourquoi ouvrir mon auto entreprise ?}). Cela vous permettra d'émettre des factures pour le travail que vous aurez réalisé.