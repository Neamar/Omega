Un travail mal fait n’est dans l’intérêt d’aucun des intervenants du site.
Les élèves garderont une mauvaise image du site, ne le réutiliseront pas et ne paieront donc plus les correcteurs. De plus, ils lanceront une procédure de \doc[/eleve/contestation]{contestation} qui pourrait empêcher le paiement du rédacteur.
Les correcteurs peuvent voir \doc[bloque]{leur compte bloqué} pour négligence.
Les administrateurs doivent vérifier la cohérence des remarques et servir de médiateurs dans un conflit qui nuit à la réputation du site.

En conséquence, eDevoir ne tolère pas les personnes fournissant un travail bâclé ou incomplet. En tant que correcteur, avant de rendre l’exercice, placez-vous dans la peau de l’élève et demandez-vous si vous aurez plaisir à lire votre texte et que vous en comprendriez les nuances. Si ce n’est pas le cas, il y a peut-être quelque chose à revoir...