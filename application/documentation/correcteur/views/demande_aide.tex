Un exercice "pistes de résolutions" devra répondre aux exigences suivantes :
\item ne contiendra pas les \textbf{réponses} aux questions, c'est-à-dire le résultat final ;
\item contiendra toutes les explications permettant à l'élève d'atteindre le bon résultat ;
\item contiendra des explications claires et détaillées à l'attention de l'élève pour que celui-ci comprenne bien la démarche choisie et les outils utilisés pour arriver au terme de l'exercice.

Il est plus que conseillé d'inclure des images illustratives dans le rendu final ainsi qu'éventuellement des liens externes vers des exemples.