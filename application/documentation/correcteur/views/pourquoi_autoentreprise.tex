Cette page explique le "pourquoi" de la création d'auto-entreprise. Pour le "comment", consultez \doc[correcteur/comment_autoentreprise]{cette autre page}.

Afin de pouvoir toucher de l'argent de la part d'\eDevoir, il faut que vous soyez une personne morale en mesure de nous établir une facture. Cette facture est générée automatiquement sur le site, vous n'avez rien à faire. Mais sur celle-ci doit être fait mention du numéro \b{siret} de l'auto-entrepreneur (ou de l'entreprise individuelle) : vous. Sans ce numéro, votre travail s'apparenterait à du travail dit "au noir". Lorsque vous effectuerez un travail, nous vous paierons et en échange vous nous établirez une facture attestant du travail réalisé. Le travail d'un auto-entrepreneur n'est pas assujetti à la TVA, par conséquent la somme que vous demanderez pour un travail sera exactement la somme que nous vous reverserons si le travail a été correctement réalisé.
Les démarches pour ouvrir votre auto-entreprise sont extrêmement simples et non contraignantes (voir \doc[correcteur/comment_autoentreprise]{comment ouvrir mon auto-entreprise ?}) : elles se font de chez vous, en moins de 10 minutes. Elles ne vous engagent absolument pas à honorer une certaine entrée d'argent.