Cette page explique le "comment" de la création d'auto-entreprise. Pour le "pourquoi", consultez \doc[correcteur/pourquoi_autoentreprise]{cette autre page}.

Les démarches pour l'ouverture de l'auto-entreprise peuvent se faire directement de chez vous, sans même bouger de votre chaise.
Elles ne sont pas contraignantes, et l'ouverture d'une auto-entreprise ne vous engage à rien pour le futur. Vous pourrez la fermer dans le futur par simple envoi d'une lettre recommandée.

\b{À noter} : si l'inscription en ligne prend moins d'une demi-heure, la réception du numéro de SIRET peut prendre jusqu'à trois semaines de délai. En conséquence, vous pouvez vous inscrire sans nous fournir pour l'instant de numéro de SIRET : vous le mettrez à jour une fois reçu. \b{Attention !} Vous ne recevrez aucun paiement tant que vous ne nous aurez pas communiqué votre numéro de SIRET.
Rendez-vous sur la \l[https://www.cfe.urssaf.fr/autoentrepreneur/CFE_Declaration]{page de déclaration auto-entrepreneur} et remplissez les différentes zones. Quelques informations utiles pour remplir sereinement le formulaire :
\begin{itemize}
	\li 1
	\li 2
	\li (à faire)
\end{itemize}

Une fois votre déclaration effectuée, vous devriez recevoir un mail de cfe\.pl-et-assoc@urssaf\.fr vous indiquant la réception de votre dossier de déclaration de création d'entreprise individuelle.
Vous pouvez alors créer votre compte sur notre site : vous le mettrez à jour via \l[/correcteur/options/]{les options} (\doc[correcteur/options]{aide}) à la réception de votre numéro de SIRET, par la poste, d'ici une quinzaine de jours. \b{Vous ne pourrez demander aucun paiement tant que vous ne nous aurez pas indiqué ce numéro} !