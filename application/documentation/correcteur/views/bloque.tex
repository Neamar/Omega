Votre compte peut être bloqué pour de multiples raisons (liste non exhaustive)
\begin{itemize}
\li Une \doc[correcteur/alerte]{alerte vie privée} a été émise. Vous avez donné votre nom ou une information personnelle en acceptant une offre, dans la FAQ...
\li Votre langage dans une \doc[correcteur/faq]{FAQ} n'est pas approprié (l'emploi d'expressions telles que "kikou" ou "lol" est strictement prohibé).
\li Vous n'avez pas envoyé dans les temps la correction d'un exercice.
\li Vous avez mal fait un exercice et les administrateurs ont jugé que les erreurs commises étaient trops graves pour vous laisser continuer à travailler avec \eDevoir.
\li Vous réservez en permanence des exercices avec des prix fantaisistes.
\li etc.
\end{itemize}

\subsection{Qu'implique le blocage ?}
Une fois bloqué, vous ne pouvez plus répondre à un appel d'offres. Le marché aux exercices vous est fermé, et seul les exercices en cours peuvent être terminés.

\subsection{Que faire ?}
Si vous êtes bloqué, il n'y a qu'une seule chose à faire : nous \doc[contact]{contacter} en vous justifiant des faits reprochés. Si nous estimons vos explications pertinentes, nous vous débloquerons. Dans le cas contraire, vous resterez bloqué indéfiniment -- tout en ayant bien sûr la possibilité de retirer les fonds disponibles sur votre compte.