\begin{enumerate}
	\li \b{Le correcteur s'inscrit.} Pour cela, il doit indiquer un certain nombre de renseignements personnels  (Identité, adresse mail et postale, téléphone, numéro SIRET, upload d'un CV au format pdf...). Nous effectuons alors une sélection stricte pour vérifier le profil et les compétences de la personne. Une fois notre décision prise, la personne rejoint le groupe des correcteurs.
	\li\b{Le correcteur renseigne ses compétences.} Il indique alors ses capacités pour que nous lui proposions les exercices les plus adaptés à ses possibilités.
	\li \b{ Le correcteur accède à la foire aux exercices.} Il s'agit d'un lieu virtuel dans lequel se trouvent tous les exercices qui n'ont pas encore été attribués et qui correspondent aux compétences du correcteur.
	\li \b{Le correcteur fait une offre.}  Cette offre est dimensionnée par rapport au devoir soumis et aux contraintes qui lui sont associées (temps avant échéance, durée de travail estimé...) Si l'élève refuse l'offre, le correcteur en est informé et ne pourra plus faire d'offre sur cet exercice.
	\li \b{Le correcteur est informé de l'assentiment de l'élève.} Il peut dès lors commencer à travailler.
	\li \b{Le correcteur effectue le travail.} Lorsque le travail est terminé, il l'envoie à l'élève via sa console d'administration sur le site.
	\li \b{Paiement du correcteur.} Au plus tard deux semaines après la deadline élève, et si aucune réclamation de la part de ce dernier n'a été émise, le correcteur reçoit les points pour le travail réalisé. En fait, ces points peuvent être envoyés au correcteur dès lors que l'élève confirme que le travail a bien été réalisé et qu'il est en adéquation avec sa demande. Mais si celui-ci ne le fait pas, nous nous engageons à envoyer au plus tard deux semaines après la deadline élève
\end{enumerate}