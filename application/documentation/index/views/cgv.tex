edevoir.com est un site édité par la société \eDevoir.

L'activité du site est de de faire l'intermédiaire en achats et ventes de documents -- scolaires et étudiants -- personnalisés, dans des domaines pluridisciplinaires. Ces documents peuvent avoir diverses formes.

Les différents termes utilisés dans les présentes conditions sont définis dans les \textit{Conditions Générales d'Utilisation}.

\section{Demande de document}

L'utilisateur fait une demande au site edevoir.com d'un document personnalisé. Le site se charge ensuite de relayer cette demande à des utilisateurs pouvant y répondre.

\section{Tarif du document}

Le tarif du document est exprimé en Points, monnaie virtuelle utilisée par \eDevoir. L'Utilisateur peut acheter des Points par divers moyens de paiement mis à disposition sur le site. Les coûts inhérents au mode de paiement ou de retrait choisi sont répercuté au moment de la commande. L'Utilisateur est conscient, au moment d'acheter des Points, que la société \eDevoir ne garantit pas de les racheter au même prix et que celle-ci est tout à fait libre de fixer ses tarifs sans aucune nécessité de notification préalable.
La demande est enregistrée à compter de la réception du paiement -- en cas de paiement par chèque -- ou de la confirmation de paiement -- en cas de paiement par carte bancaire ou appel surtaxé par exemple.

\section{Achat du document}

Pour acheter un document , l'utilisateur devra remplir au préalable un formulaire d'inscription au site. Il aura le choix pour remplir son compte en Points entre plusieurs systèmes de paiement en ligne (paiement par carte bleue, système Paypal, appels surtaxés etc.).

\section{Téléchargement du document}

Les documents proposés par le Site sont téléchargeables dès leur mise à disposition sur le site. Une notification par email peut alors être envoyé à l'utilisateur concerné.

\section{Droit de rétractation}

Conformément à l'article L 121-20-2 du code de la consommation, l'acheteur ne bénéficie pas du droit de rétractation. 

\section{Responsabilités de l'acheteur}

Lors de la commande sur le site d'un document, l'utilisateur certifie :
\begin{enumerate}
   \item Être âgé d'au moins 18 ans ou, à défaut, d'avoir au préalable obtenu l'accord de son parent ou tuteur légal.
   \item Avoir compris et accepté les \textit{Conditions Générales d'Utilisation} (disponibles sur le site edevoir.com).
   \item Être le propriétaire du moyen de paiement utilisé (carte bancaire, forfait téléphonique etc.) ou avoir l'accord de son propriétaire légitime.
   \item Être conscient et accepter que le contenu vendu par \eDevoir n'a qu'un but éducatif. \eDevoir interdit la reproduction, la copie ou le plagiat du contenu que vous achetez sur le site edevoir.com
   \item Être conscient et accepter que le site edevoir.com ne certifie par l'exactitude du contenu dudit document.
   \item Être conscient et accepter que edevoir.com est susceptible d'annuler une transaction de manière unilatérale sans que ceci ait à être justifié ou ne fasse l'objet d'un dédommagement quelconque.
   \item Être conscient que le contenu dudit document est protégé par copyright et ne peut ni être vendu ni transféré à une tierce personne.
   \item Être conscient que la société \eDevoir met tout en oeuvre pour assurer la sécurité des informations personnelles et bancaires envoyées sur son site, mais que sa responsabilité ne saurait être engagée en cas de non respect des \textit{Conditions Générales d'Utilisation} par l'utilisateur, d'un problème technique ou d'un attaque pirate.
   \item Être conscient que le document commandé ne saurait faire l'objet d'un remboursement ou d'une rétractation.
\end{enumerate}
