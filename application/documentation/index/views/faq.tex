\section{Mais c'est quoi \eDevoir ?}
\eDevoir est un service en ligne réalisant des corrigés ou des aides pour des problèmes scolaires.

\section{C'est cher ?}
Ça dépend de ce que vous demandez ! Le tarif est fixé selon la demande. Plus elle est simple, moins vous payez.
Encore plus subtil : vous pouvez refuser les offres qui vous sont faites en espérant obtenir mieux auprès d'une autre personne ! Bilan : vous contrôlez parfaitement vos coûts.

\section{Quels sont les délais ?}
Les vôtres ! Vous nous indiquez votre date limite pour le rendu, et le correcteur sera tenu de respecter cette contrainte. S'il échoue, \doc[eleve/retard]{vous êtes remboursé à 200\%} !

\section{Et si le travail est mal fait ?}
Il vous suffit de nous le dire. Nous examinerons alors votre demande pour procéder \doc[eleve/contestation]{à un remboursement} au double du préjudice subi.

\section{Quels sont les moyens de paiement ?}
\eDevoir utilise un système de points servant de monnaie virtuelle.
Plusieurs méthodes sont disponibles pour transformer de l'argent en points :
\item Par SMS surtaxé
\item Par Paypal
\item Par virement bancaire (bientôt) !