\begin{enumerate}
\li \b{L'élève poste un devoir}. Il indique la matière, le niveau, le délai, le type de l'exercice (QCM ? Corrigé d'interrogation ratée ?) et précise sa demande (Corrigé complet / Pistes de correction). Le support de l'exercice peut être un scan, un texte ou une photo.
\li \b{Diffusion de la demande}. L'exercice posté par l'élève est transmis aux correcteurs compétents dans la matière correspondante.
\li \b{Le correcteur fait une offre}.  Lorsque l'offre est envoyée à l'élève, l'exercice n'est plus disponible dans la liste pour les autres correcteurs. Cette offre est dimensionnée par rapport au devoir soumis et aux contraintes qui lui sont associées (temps avant échéance, durée de travail estimée...)
\li \b{L'élève reçoit une offre}. Il peut la refuser, auquel cas une nouvelle offre lui sera transmise (sans garantie de baisse des prix).
\li \b{L'élève accepte l'offre}. Quand l'offre convient, l'élève l'accepte. Il peut alors vaquer à ses autres occupations intellectuelles, nous travaillons pour lui !
\li \b{Le correcteur est informé de l'assentiment de l'élève}. Il peut dès lors commencer à travailler.
\end{enumerate}