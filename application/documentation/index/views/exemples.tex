Énoncé :
Énoncer puis démontrer les formules d'addition en trigonométrie

Soit un cercle trigonométrique dans le repère $(O, \vec{i}, \vec{j})$.
Soient trois points $M$, $M'$ et $N$ du cercle tels que :
\begin{itemize}
	\li $M$ tel que $(\vec{OI}, \vec{OM}) = a\,(2\pi)$
	\li $M'$ tel que $(\vec{OI}, \vec{OM'}) = a + b\,(2\pi)$
	\li $N$ tel que $(\vec{OM}, \vec{ON}) = \frac{\pi}{2}\,(2\pi)$
\end{itemize}

Le point $M'$ dans le répère orthonormal $(O, \vec{OM}, \vec{ON})$ est tel que $(\vec{OM}, \vec{OM'}) = b\,(2\pi)$.
Donc :
$$\tag{1} \vec{OM'} = \cos(b)\vec{OM} + \sin(b)\vec{ON}$$
Or dans le repère $(O, \vec{i}, \vec{j})$ on a :
\begin{itemize}
	\li $\vec{OM} = \cos(a)\vec{i} + \sin(a)\vec{j}$
	\li $\vec{ON} = \cos(a + \frac{\pi}{2})\vec{i} + \sin(a + \frac{\pi}{2})\vec{j} = -\sin(a)\vec{i} + \cos(a)\vec{j}$ car $\cos(a + \frac{\pi}{2}) = -\sin(a)$ et $\sin(a + \frac{\pi}{2}) = \cos(a)$.
\end{itemize}

$\begin{eqnarray}(1) & \Leftrightarrow & \vec{OM} = \cos(b) \times (\cos(a)\vec{i} + \sin(a)\vec{j}) + \sin(b)(-\sin(a)\vec{i} + \cos(a))\vec{j} \\ & \Leftrightarrow & \color{green}{(\cos(a)\cos(b) - \sin(a)\sin(b))}\vec{i} + \color{blue}{(\sin(a)\cos(b) + \sin(b)\cos(a))}\vec{j}\end{eqnarray}$

Or dans le repère orthonormal $(O, \vec{i}, \vec{j})$,
$\vec{OM'}$ a pour coordonnées $\color{green}{\cos(a + b)}\vec{i} + \color{blue}{\sin(a+b)}\vec{j}$.
Par identification, on en déduit donc les relations cherchées :
$$\cos(a + b) = \cos(a)\cos(b) - \sin(a)\sin(b)$$
$$\sin(a + b) = \sin(a)\cos(b) + \sin(b)\cos(a)$$