Le site eDevoir fonctionne avec un système de points -- ou d'argent virtuel -- que vous soyez élève ou correcteur : 1€ équivaut à \b{_EQUIVALENCE_POINT_ points}.
\item \b{Vous êtes élève}. Lorsque vous commandez un exercice, les propositions seront faites en points. \b{Attention} : lorsque vous déposez de l'argent sur votre compte \b{Devoir} c'est l'argent effectivement perçu qui est converti en points. Les frais inhérents à Paypal ou Allopass par exemple sont à votre charge ; aucun frais à prévoir si vous réglez par carte bancaire. Toutes ces modalités sont développées sur la \doc[eleve/depot]{page d'aide au dépôt}.
\item \b{Vous êtes correcteur}. Lorsque vous transmettez une proposition à un élève, vous le faites en points. Idem lorsque vous envoyez un exercice (qu'il n'a pas fait l'objet d'une \doc[eleve/contestation]{réclamation}) la somme convenue pour l'exercice est créditée sur votre compte eDevoir sous forme de points que vous pourrez \doc[correcteur/retrait]{retirer à tout moment}.