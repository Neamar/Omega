Si un élève accepte une offre, une FAQ s'ouvre automatiquement (consulter \doc[eleve/faq]{cette page pour plus d'information élève}, \doc[correcteur/faq]{celle-ci pour plus d'information correcteur}).

Les éléments de cette FAQ (question ou réponse) permettent l'insertion de formules mathématiques.
Toutes les formules sont indiquées entre \$. Elles seront automatiquement mises en forme après avoir posté l'élément dans la FAQ.

\begin[|texable]{tabular}
\caption Liste des structures usuelles
\head
Code & Rendu
\body
\frac{x}{y}
\int_{x}^{y}
\sum_{x}^{y}
\prod_{x}^{y}
\sqrt{x}
\sqrt[y]{x}
\overrigtharrow{AB}
\mathbb{R}
x_{y}
x^{y}
\end{tabular}

\begin{tabular}
\caption Liste des symboles
\head
Code & Rendu
\body
\alpha
\beta
…
\Alpha
\Beta
...
\leq
\geq
\simeq
\longrightarrow
\Rigtharrow
\Leftrigtharrow
\forall
\exists
\infty
\emptyset
\end{tabular}

\begin{tabular}
\caption Liste des environnements
\head
Code & Rendu
\body

\left\{\begin{matrix} 4x + 2y = -1 \\ 3x - y = 2 \end{matrix}\right\}
ou
\left[\matrix{2 & 4 \cr 0 & 5 \cr 1 & 9 \cr 0 & 0}\right]

\left{
\begin{eqnarray*}
x=1 \\
y=2 \\
z=3 \\
\end{eqnarray*}
\right}



\end{tabular}