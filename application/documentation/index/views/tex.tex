Si un élève accepte une offre, une FAQ s'ouvre automatiquement (consulter \doc[eleve/faq]{cette page pour plus d'information élève}, \doc[correcteur/faq]{celle-ci pour plus d'information correcteur}).

Les éléments de cette FAQ (question ou réponse) permettent l'insertion de formules mathématiques.
Toutes les formules sont indiquées entre \$. Elles seront automatiquement mises en forme après avoir posté l'élément dans la FAQ.

\begin[texable]{tabular}
\caption Liste des structures usuelles. 
\head
Type & Rendu.
\body
Fraction & $$\frac{x}{y}$$ \\
Intégrale & $$\int_{x}^{y}$$ \\
Somme & $$\sum_{x}^{y}$$ \\
Produit & $$\prod_{x}^{y}$$ \\
Racine carrée & $$\sqrt{x}$$ \\
Racine n-ième & $$\sqrt[y]{x}$$ \\
Vecteur & $$\vec{AB}$$ \\
Ensemble & $$\mathbb{R}$$ \\
Indice & $$x_{y}$$ \\
Exposant & $$x^{y}$$ \\
\end{tabular}

\begin[texable]{tabular}
\caption Liste des symboles
\head
Rendu
\body
$$\alpha$$ \\
$$\beta$$ \\
... \\
$$\Alpha$$ \\
$$\Beta$$ \\
... \\
$$\leq$$ \\
$$\geq$$ \\
$$\simeq$$ \\
$$\longrightarrow$$ \\
$$\Rigtharrow$$ \\
$$\Leftrigtharrow$$ \\
$$\forall$$ \\
$$\exists$$ \\
$$\infty$$ \\
$$\emptyset$$ \\
\end{tabular}

\begin[texable]{tabular}
\caption Liste des environnements
\head
Rendu
\body

$$\left\{\begin{matrix} 4x + 2y = -1 \\ 3x - y = 2 \end{matrix}\right\}$$ \\
$$\left[\matrix{2 & 4 \cr 0 & 5 \cr 1 & 9 \cr 0 & 0}\right]$$ \\
$$\left{
\begin{eqnarray*}
x=1 \\
y=2 \\
z=3 \\
\end{eqnarray*}$$
\right}



\end{tabular}