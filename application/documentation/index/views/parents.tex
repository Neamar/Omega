On est tous passés par là : notre enfant scolarisé en terminale S demandant ce qu'est un barycentre ou comment dériver une suite géométrique. Quoi de plus normal que de rester sans voix face à une telle demande ? Il est loin le temps où il vous demandait "Comment on fait les bébés"...

\b{eDevoir.com} se veut une alternative rapide et efficace face des obligations scolaires parfois inadaptées, mais aussi une aide précieuse pour des parents désireux d'aider leur(s) enfant(s). En effet, vous, parents, n'êtes pas forcément aptes à le faire.

Bien entendu, ce corrigé -- réalisé dans les délais que vous nous imposez -- est d'une qualité impeccable (vous pouvez consulter \doc[exemples]{nos exemples}) et adapté à vos besoins. Il est également possible de réaliser pour votre enfant un document le guidant précisément à travers les étapes de résolution de son exercice -- sans lui apporter toutes les réponses sur un plateau.

Quelle que soit la formule que vous choisirez, vous disposez après l'envoi du corrigé d'un suivi personnalisé avec le correcteur pour lui poser toutes les questions que vous pourriez encore avoir sur votre exercice.

\l[/eleves/inscription]{Cliquez pour commencer !}