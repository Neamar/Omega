Chez \eDevoir, nous prenons très à cœur votre vie privée.

\b{Élèves}, \b{parents}, nous ne voulons pas vous imposer de contraintes ! Vous pouvez vous inscrire sur notre site dans l'anonymat le plus complet, sans avoir à rendre de comptes à qui que ce soit. Votre seul lien avec nous est l'adresse mail que vous nous fournirez lors de votre inscription. À tout moment, vous pouvez fermer votre espace membre : plus personne n'y aura accès et nous ne conserverons que le minimum légalement requis.

\b{Correcteurs}, nous vous demandons plus d'informations afin de garantir un travail de qualité.  Nom, prénom, adresse, téléphone et mail sont des données obligatoires : en cas de défaut, de réclamation ou de remarque, nous prenons toutes les dispositions pour assurer qu'un tel incident ne se reproduise pas et ne vienne pas entâcher l'expérience de nos clients.

Dans les deux cas, nous ne fournirons pas de données personnelles ou bancaires à des tiers (sauf obligations légales). Ces informations sont privées et le resteront.

\eDevoir prend très au sérieux la protection de vos données personnelles et se conforme strictement aux règles de la vie privée. Les informations personnelles ne sont sur ce site qu'à des fins techniques. En aucun cas, ces données ne pourront être vendues ni cédées à un tiers.

Conformément à la loi (voir ce qu'il est légal de faire concernant le stockage des données)