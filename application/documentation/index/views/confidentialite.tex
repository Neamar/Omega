\b{\eDevoir}
SARL au capital de 1000€
530 684 950 RCS LYON - N\up{o} SIRET : 530 684 950 00013
Siège social :
\eDevoir
144 rue du Dauphiné\\
69003 LYON\\
e-mail : contact@edevoir.com

Chez \eDevoir, nous prenons très à cœur votre vie privée.

\section{Objet de cette politique de confidentialité du site}

Cette politique de confidentialité du site porte sur le traitement des renseignements personnels d'identification qui sont recueillis lorsque vous êtes sur le site web et lorsque vous utilisez les services du site web, comme décrit dans cette politique de confidentialité du site.

Cette politique ne s'applique pas aux pratiques de sociétés que \eDevoir ne possède ou ne contrôle pas, ni aux personnes que \eDevoir. \eDevoir n'est pas responsable des politiques ou pratiques en matière de respect de la vie privée des autres sites web auxquels vous choisissez de vous connecter à partir du site web.

\section{Recueil et utilisation des renseignements}

\eDevoir ne recueille pas de renseignements personnels d'identification lorsque vous accédez à la page d'accueil du site web et que vous naviguez sur le site web. Certaines informations vous serons cependant demandées au moment de votre inscription sur le site et au moment de votre identification sur celui-ci.

\b{Élèves}, \b{parents}, nous ne voulons pas vous imposer de contraintes ! Vous pouvez vous inscrire sur notre site dans l'anonymat le plus complet, sans avoir à rendre de comptes à qui que ce soit. Votre seul lien avec nous est l'adresse mail que vous nous fournirez lors de votre inscription. À tout moment, vous pouvez fermer votre espace membre : plus personne n'y aura accès et nous ne conserverons que le minimum légalement requis.

\b{Correcteurs}, nous vous demandons plus d'informations afin de garantir un travail de qualité.  Nom, prénom, adresse, téléphone et mail sont des données obligatoires : en cas de défaut, de réclamation ou de remarque, nous prenons toutes les dispositions pour assurer qu'un tel incident ne se reproduise pas et ne vienne pas entacher l'expérience de nos clients.

Dans les deux cas, nous ne fournirons pas de données personnelles ou bancaires à des tiers (sauf obligations légales). Ces informations sont privées et le resteront.

\eDevoir prend très au sérieux la protection de vos données personnelles et se conforme strictement aux règles de la vie privée. Les informations personnelles ne sont sur ce site qu'à des fins techniques. En aucun cas, ces données ne pourront être vendues ni cédées à un tiers.

\section{Échange et communication de renseignements}

\eDevoir peut envoyer des renseignements personnels d'identification à votre sujet à d'autres sociétés ou personnes lorsque:
\begin{itemize}
	\li nous avons votre consentement pour échanger ces renseignements; ou
	\li nous avons besoin d'échanger vos renseignements pour fournir le produit ou service que vous avez demandé; ou
	\li nous fournissons les renseignements à des sociétés qui travaillent pour notre compte selon des accords de confidentialité; ou
	\li nous répondons à des ordonnances de production de pièces, des ordonnances de tribunaux ou à un acte de procédure; ou
	\li nous pensons qu'il est nécessaire, comme nous l'avons établi en toute liberté, d'enquêter, de prévenir ou de prendre des mesures concernant des activités illégales, une escroquerie présumée, des situations d'urgence comportant des menaces potentielles sur la sécurité physique de toute personne, des violations des conditions d'utilisation de edevoir.com, ou comme l'exige la loi par ailleurs.
\end{itemize}

\section{Sécurité}

Nous avons mis en œuvre des mesures techniques et organisationnelles destinées à protéger vos renseignements personnels contre une perte accidentelle, un accès et une utilisation non autorisés, l'altération ou la divulgation. Toutefois, Internet est un système ouvert et nous ne pouvons garantir que des tiers non autorisés ne seront jamais capables de neutraliser ces mesures ou d'utiliser vos renseignements personnels à des fins illicites.

Si vous communiquez avec \eDevoir par courriel, vous devez noter que le secret du courriel par Internet est incertain. En envoyant par courrier électronique des messages ou des informations sensibles ou confidentiels qui ne sont pas cryptés, vous acceptez le risque de cette incertitude et le manque éventuel de confidentialité sur Internet.

\section{Changements dans cette politique de confidentialité du site.}

Cette politique peut être modifiée en cas de besoin. Si nous modifions la politique de confidentialité du site de quelque manière que ce soit, nous mettrons une version mise à jour sur cette page du site web. Le fait d'examiner régulièrement cette page garantit que vous serez toujours au courant des renseignements que nous recueillons, de la manière dont nous les utilisons et dans quelles circonstances, s'il y en a, nous les échangerons avec d'autres parties. Si nous apportons des changements substantiels dans la manière dont nous utilisons vos renseignements personnels, nous vous en informerons en diffusant une annonce bien mise en valeur sur le site web.
