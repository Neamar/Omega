\section{Prérequis}

Les présentes Conditions Générales ont pour objet de définir les modalités de mise à disposition des services du site edevoir.com, ci-après nommé «~le Service~» et les conditions d'utilisation du Service par l'Utilisateur.

Tout accès et/ou utilisation du site edevoir.com suppose l'acceptation et le respect de l'ensemble des termes des présentes Conditions et leur acceptation inconditionnelle. Elles constituent donc \textbf{un contrat entre \eDevoir et l'Utilisateur}.

Dans le cas où l'Utilisateur ne souhaite pas accepter tout ou partie des présentes conditions générales, il lui est demandé de renoncer à tout usage du Service.

edevoir.com est un site édité par la société \textbf{\eDevoir}.

\paragraph*{\eDevoir}
SARL au capital de 1~000\euro\\
RCS LYON - N\up{o} SIRET : [en cours d'attribution]\\
Siège social :\\
\eDevoir\\
144 rue du Dauphiné\\
69003 LYON\\
e-mail : contact@edevoir.com

\paragraph*{Hébergeur :}
OVH\\
2 rue Kellermann\\
59100 ROUBAIX

\paragraph*{Informatique et Libertés :}
conformément à la loi du 6 janvier 1978 relative à l'informatique, aux fichiers et aux libertés, edevoir.com a fait l'objet d'une déclaration auprès de la Commission Nationale Informatique et Libertés, numéro : [en cours d'attribution]

\paragraph*{Données personnelles :}
vous disposez d'un droit d'accès, de modification, de rectification et de suppression des données qui vous concernent (art. 34 de la loi "Informatique et Libertés" du 6 janvier 1978).\\
Pour exercer ce droit, envoyez un courrier au siège social de l'entreprise ou envoyez un mail à contact@edevoir.com


\section{Définitions}

\begin{itemize}
  \item \textbf{edevoir.com} : site internet édité par la société \eDevoir qui offre différents services visant à offrir à l'\'Elève un accompagnement de sa scolarité dont il constitue le complément et à laquelle il ne saurait en aucune circonstance se substituer.

  \item \textbf{Utilisateur} : l'« Utilisateur » est toute personne qui utilise le Site ou l'un des services proposés sur le Site.

  \item \textbf{Administrateur} : le terme « Administrateur » désigne l'utilisateur qui est en plus autorisé à assurer la gestion du Site, c'est-à-dire qui porte la responsabilité de l'intégrité du système et de sa bonne marche.
  
  \item \textbf{Contenu Utilisateur} : le terme « Contenu Utilisateur » désigne les données transmises par l'Utilisateur dans les différentes rubriques du Site.

  \item \textbf{Membre} : le terme « Membre » désigne un utilisateur identifié sur le site.

  \item \textbf{\'Elève} : le terme « \'Elève » désigne toute personne qui s'inscrit sur le Site du côté dit \textit{\'Elève}.

  \item \textbf{Correcteur} : le terme « Correcteur » désigne toute personne qui s'inscrit sur le Site du côté dit \textit{Correcteur}.
  
  \item \textbf{Point} : le terme « Point » désigne la monnaie virtuelle utilisée par le site edevoir.com. Un Point peut être acheté ou revendu sur le site. Les coûts inhérents au mode de paiement ou de retrait choisi sont répercuté au moment de la commande.

  \item \textbf{Identifiant} : le terme « Identifiant » recouvre les informations nécessaires à l'identification d'un utilisateur sur le site pour accéder aux zones réservées aux membres. Sur edevoir.com, l'Identifiant est l'adresse mail de l'Utilisateur.

  \item \textbf{Mot de passe} : Le « Mot de passe » est une information confidentielle, dont l'Utilisateur doit garder le secret, lui permettant, utilisé conjointement avec son Identifiant, de prouver son identité.
\end{itemize}


\section{Accès au service}

Le Service est accessible gratuitement à tout Utilisateur disposant d'un accès à internet. Tous les coûts afférents à l'accès au Service, que ce soient les frais matériels, logiciels ou d'accès à internet sont exclusivement à la charge de l'utilisateur. Il est seul responsable du bon fonctionnement de son équipement informatique ainsi que de son accès à internet.

Certaines sections du site sont réservées aux Membres après identification à l'aide de leur Identifiant et de leur Mot de passe.
Le site edevoir.com se réserve le droit de refuser l'accès au Service, unilatéralement et sans notification préalable, à tout Utilisateur ne respectant pas les présentes conditions d'utilisation.

\eDevoir met en œuvre tous les moyens raisonnables à sa disposition pour assurer un accès de qualité au Service, mais n'est tenu à aucune obligation d'y parvenir.

\eDevoir ne peut, en outre, être tenu responsable de tout dysfonctionnement du réseau ou des serveurs ou de tout autre événement échappant au contrôle raisonnable, qui empêcherait ou dégraderait l'accès au Service.

\eDevoir se réserve la possibilité d'interrompre, de suspendre momentanément ou de modifier sans préavis l'accès à tout ou partie du Service, afin d'en assurer la maintenance, ou pour toute autre raison, sans que l'interruption n'ouvre droit à aucune obligation ni indemnisation.


\section{Le fonctionnement par Points}

Chaque demande de correction est facturée à l’\'Elève en Points. Celui-ci peut acheter des Points par divers moyens de paiement mis à disposition sur le site. De même, chaque correction apportée par un Correcteur est échangée contre un certain nombre de Points, qui sera fixé librement par \eDevoir et dans un délais maximum d'un mois suivant la limite de temps imposée par l'\'Elève. Le prix d'un Point est susceptible de varier au cours du temps et est fonction du mode de paiement choisi. L'Utilisateur est conscient, au moment d'acheter des Points, que la société \eDevoir ne garantit pas de les racheter au même prix et que celle-ci est tout à fait libre de fixer ses tarifs sans aucune nécessité de notification préalable.
La demande de correction est enregistrée à compter de la réception du paiement -- en cas de paiement par chèque -- ou de la confirmation de paiement -- en cas de paiement par carte bancaire ou par appel surtaxé par exemple.

Elle est traitée dans les délais demandés, sauf défaillance technique ou cas de force majeure.

L'Utilisateur-Correcteur ne pourra revendre des Points à \eDevoir tant qu'il n'aura pas indiqué sur le site son numéro de SIRET, lui permettant légalement d'établir une facture.


\section{Obligations de l'utilisateur}

Le site edevoir.com ne peut être consulté par l'Utilisateur (\'Elève) qu'à titre personnel, à des seules fins pédagogiques et éducatives, dans le cadre de l'inscription réalisée par lui à cet effet et dans le respect des présentes conditions. Les corrections et éléments de corrections apportés aux devoirs qu’il soumet à edevoir.com ne sont en aucun cas destinés à être utilisés telles quelles dans un cadre scolaire, à des fins de triche notamment.

L'Utilisateur se doit d'être majeur au moment d'acheter des Points sur le site edevoir.com ou, à défaut, d'avoir obtenu au préalable l'autorisation de son parent ou tuteur légal. La responsabilité de \eDevoir ne saurait être engagée si un mineur utilise, à l'insu de son parent ou tuteur légal, un moyen de paiement (appel surtaxé par exemple) sans accord de la personne. Il incombe à chaque personne responsable de ne pas laisser à libre disposition d'autres personnes des moyens de paiement personnels.

En outre, \eDevoir ne concède à l'Utilisateur aucune licence ni aucun autre droit que celui de consulter le site ainsi que les pistes et éléments de correction apportés aux devoirs qu'il soumet aux fins susvisées.

L'Utilisateur s'interdit :
\begin{enumerate}
   \item de diffuser le contenu du site par quelque moyen que ce soit ;
   \item de télécharger le contenu du site autrement que dans les cas autorisés par \eDevoir et, plus généralement, de le fixer et reproduire sur tout support quel qu'il soit, par quelque procédé que ce soit, à la seule exception d'une impression sur support papier dans les limites ci-après : en cas d'impression sur support papier, l'Utilisateur garantit que toutes les mentions figurant éventuellement sur le contenu ainsi imprimé relatives à la protection des droits seront reproduites sans modifications. Ces copies ne pourront être utilisées que par le seul Utilisateur à des fins personnelles pour son seul usage privé à l'exclusion de tout autre et ne pourront être diffusées auprès de tiers qu'après autorisation expresse de \eDevoir ;
   \item plus généralement d'exploiter et/ou d'utiliser tout ou une partie des éléments du site par quelque moyen que ce soit et sous quelque forme que ce soit à des fins autres que celles expressément autorisées ;
   \item de se créer plusieurs comptes. L'utilisation de plusieurs comptes pour une personne unique pourra entrainer le bannissement définitif et totalement du site ;
   \item de violer les présentes conditions. Dans le cas contraire, l'Utilisateur sera susceptible de s'exposer à des sanctions pénales et pourra se voir demander des dommages et intérêts à hauteur du préjudice subi.
\end{enumerate}


\section{Règles générales de rédaction}

Il est interdit :
\begin{itemize}
  \item de diffuser des informations personnelles de quelque nature que ce soit : nom, prénom, numéro de téléphone, adresse mail (liste non exhaustive) ;
  \item d'insérer des mots-clés dans les textes ;
  \item de diffuser toute forme de publicité ou de spam (message non sollicité) ;
  \item d'insérer dans un texte un contenu jugé abusif (apologie de crimes contre l'humanité, incitation à la haine raciale, pornographie, atteinte aux droits d'auteur, atteinte aux droits des marques, atteinte au droit à l'image). De manière générale, les sujets à caractère illégal sont interdits et seront supprimés.
\end{itemize}

Les Administrateurs sont seuls juges. L'Utilisateur doit respecter leurs remarques, sans quoi il pourrait être banni du site.


\section{Conditions d'achat de documents}

Ne seront pas achetés les documents répondant aux critères suivants :
\begin{enumerate}
   \item Les corrections ne répondant pas \textbf{intégralement} au sujet traité.
   \item Les corrections trop courtes où le sujet n'est abordé que de façon superficielle.
   \item Les corrections contenant des fautes d'orthographe.
   \item Les corrections portant atteinte à l'ordre public et aux bonnes m\oe{}urs.
\end{enumerate}

Les Correcteurs restent seuls responsables du contenu des Documents qu'ils déposent.

\eDevoir se réserve le droit de fixer librement le prix en Points d'un document -- entre 0~Point et $n$~Points, où $n$ est le nombre de Points suggéré par le Correcteur. Cette décision sera rendue après lecture dudit document par eDevoir et par l'\'Elève, dans un délai maximal d'un mois suivant la date limite fixée par l'\'Elève. En envoyant un document sur edevoir.com, le Correcteur consent donc à ne pas être maître du prix en Points qu'il touchera en échange et s'engage à respecter la décision de \eDevoir, quelle qu'elle soit.


\section{Propriété intellectuelle}

Le site edevoir.com, notamment son contenu, est protégé par le droit en vigueur en France. \eDevoir est le titulaire exclusif de l'intégralité des droits de propriété intellectuelle sur le site et son contenu (textes, photographies, illustrations, images, logos, etc.).

Le contenu reproduit sur le Site fait l’objet d’un droit d'auteur et sa reproduction ou sa diffusion, sans autorisation expresse écrite de \eDevoir, constitue une contrefaçon passible de sanctions pénales.

« edevoir.com » est une marque déposée de \eDevoir. Toute reproduction non autorisée de cette marque, des logos et signes distinctifs constitue une contrefaçon passible de sanctions pénales. Le contrevenant s'expose à des sanctions civiles et pénales et notamment aux peines prévues aux articles L. 335.2 et L. 343.1 du code de la Propriété Intellectuelle.

L’Utilisateur est seul responsable du Contenu Utilisateur qu’il met en ligne via le Service, ainsi que des textes et/ou opinions q'’il formule.

L'Utilisateur cède expressément et gracieusement à \eDevoir tous droits de reproduction, de représentation et d'adaptation de ses contributions, pour la durée légale de protection des droits d'auteur. L'Utilisateur s'engage notamment à ce que ces données ne soient pas de nature à porter atteinte aux intérêts légitimes de tiers quels qu'ils soient. À ce titre, il garantit \eDevoir contre tous recours, fondés directement ou indirectement sur ces propos et/ou données, susceptibles d'être intentés par quiconque à l'encontre de \eDevoir. Il s'engage en particulier à prendre en charge le paiement des sommes, quelles qu'elles soient, résultant du recours d'un tiers à l'encontre de \eDevoir, y compris les honoraires d'avocat et frais de justice.

Il est interdit aux Utilisateurs, en dehors d'un usage privé, de copier, reproduire, diffuser, vendre, publier, exploiter de toute autre manière et diffuser dans un autre format sous forme électronique ou autre les informations présentes sur le site. En conséquence, toute autre utilisation est constitutive de contrefaçon et sanctionnée au titre de la Propriété Intellectuelle, sauf autorisation préalable de la société \eDevoir. Aucune reproduction, même partielle, autres que celles prévues à l'article L.122-5 du Code de la propriété intellectuelle, ne peut être faite de ce Site sans l'autorisation préalable et expresse de la société.

Si l'Utilisateur constate la présence d'un contenu jugé abusif (apologie de crimes contre l'humanité, incitation à la haine raciale, pornographie enfantine, atteinte aux droits d'auteur, atteinte aux droits des marques, atteinte au droit à l'image) ou qu'il porte atteinte aux droits d'une tierce personne, il est tenu de le signaler immédiatement par mail à l'équipe d'administrateurs (contact@edevoir.com) qui réagira le plus rapidement possible. Nous nous réservons la possibilité de retirer un contenu si nous avons des raisons de penser que celui-ci constitue une violation des présentes \textit{Conditions Générales d'Utilisation} ou de droits détenus par des tiers.

En ce sens, edevoir.com (\eDevoir) ne pourra être qualifié d'éditeur de contenu mais uniquement d'hébergeur de contenu. Seul l'Utilisateur est responsable des contenus qu'il a mis à disposition sur le Service.

\eDevoir se réserve le droit de supprimer tout ou partie du Contenu Utilisateur, à tout moment et pour quelque raison que ce soit, sans avertissement ou justification préalable. L'Utilisateur ne pourra faire valoir aucune réclamation à ce titre. 


\section{Données personnelles}

Dans une logique de respect de la vie privée de ses Utilisateurs, \eDevoir s'engage à ce que la collecte et le traitement d'informations personnelles, effectués au sein du présent site, soient effectués conformément à la loi n\up{o}78-17 du 6 janvier 1978 relative à l'informatique, aux fichiers et aux libertés, dite Loi «~Informatique et Libertés~».À ce titre, le site edevoir.com fait l'objet d'une déclaration à la CNIL [numéro en cours d'attribution].

Conformément à l'article 34 de la loi « Informatique et Libertés », \eDevoir garantit à l'Utilisateur un droit d'opposition, d'accès et de rectification sur les données nominatives le concernant. L'Utilisateur a la possibilité d'exercer ce droit :
\begin{itemize}
   \item en rentrant directement en contact par l'adresse mail suivante : contact@edevoir.com
   \item en envoyant un courrier à l'adresse suivante : \eDevoir -- 144 rue du Dauphiné -- 69003 LYON
\end{itemize}
 Dans les deux cas, veillez à bien préciser votre Identifiant afin que nous puissions vérifier que vous en êtes bien le propriétaire.

\eDevoir peut être amené à conserver et utiliser les adresses IP des utilisateurs pour une identification ultérieure par le site ou si la demande en est faite par  une autorité judiciaire, de police, ou toute autorité habilitée par la loi.

Toute collecte automatisée des données présentes sur le site est strictement interdite.


\section{Preuve, conservation et archivage}

Les registres informatisés conservés dans nos systèmes dans le respect des règles de l'art en matière de sécurité, seront considérés comme preuves des communications de courriers électroniques, envois de formulaire d'inscription, téléchargements de documents et l'envoi de commentaires. L'archivage des formulaires d'inscription est effectué sur un support de nature à assurer le caractère fidèle et durable requis par les dispositions légales en vigueur. Il est convenu qu'en cas de divergence entre nos registres informatisés et les documents au format papier ou électronique dont vous disposez, nos registres informatisés feront foi.

 
\section{Limites de responsabilités}

Le site edevoir.com est un site qui propose de l'achat et de la vente de documents -- scolaires et étudiants -- personnalisés.

Les informations vendues sur le site edevoir.com proviennent de sources réputées fiables et dont le sérieux est préalablement vérifié. Toutefois, \eDevoir ne peut garantir l'exactitude ou la pertinence des documents vendus sur son site edevoir.com. En outre, les informations vendues sur ce site le sont uniquement à titre purement informatif et ne sauraient constituer en aucun cas un conseil ou une recommandation de quelque nature que ce soit.

En conséquence, l'Utilisation des informations et contenus fournis ou vendus sur l'ensemble du site, ne sauraient en aucun cas engager la responsabilité de \eDevoir, à quelque titre que ce soit. L'Utilisateur est seul maître de la bonne utilisation, avec discernement et esprit, des informations mises à sa disposition sur le Site.

Par ailleurs, l’Utilisateur s’engage à indemniser \eDevoir de toutes conséquences dommageables liées directement ou indirectement à l'usage qu'il fait du Service.

L'accès à certaines sections du site edevoir.com nécessite l'utilisation d'un Identifiant et d'un Mot de passe. Le Mot de passe, choisi par l'utilisateur, est personnel et confidentiel. L'utilisateur s'engage à conserver secret son Mot de passe et à ne pas le divulguer sous quelque forme que ce soit. L'utilisation de son Identifiant et de son Mot de passe à travers internet se fait aux risques et périls de l'Utilisateur. Il appartient à l'Utilisateur de prendre toutes les dispositions nécessaires permettant de protéger ses propres données contre toute atteinte.

\eDevoir s'engage néanmoins à mettre en place tous les moyens nécessaires pour garantir la sécurité et la confidentialité des données transmises. L'Utilisateur est informé qu'un ou plusieurs cookies pourront être placés sur son disque dur afin d'assurer son identification.

L'Utilisateur admet connaître les limitations et contraintes propres au réseau internet et, à ce titre, reconnaît notamment l'impossibilité d'une garantie totale de la sécurisation des échanges de données. \eDevoir ne pourra pas être tenu responsable des préjudices découlant de la transmission de toute information, y compris de celle de son identifiant et/ou de son mot de passe, via le Service.

\eDevoir ne pourra en aucun cas, dans la limite du droit applicable, être tenu responsables des dommages et/ou préjudices, directs ou indirects, matériels ou immatériels, ou de quelque nature que ce soit, résultant d'une indisponibilité du Service ou de toute Utilisation du Service. Le terme « Utilisation » doit être entendu au sens large, c'est-à-dire tout usage du site quel qu'il soit, licite ou non.

L'Utilisateur s'engage, d'une manière générale, à respecter l'ensemble de la réglementation en vigueur en France.


\section{Liens hypertextes}

edevoir.com propose des liens hypertextes vers des sites web édités et/ou gérés par des tiers.

Dans la mesure où aucun contrôle n'est exercé sur ces ressources externes, l'Utilisateur reconnaît que \eDevoir n'assume aucune responsabilité relative à la mise à disposition de ces ressources, et ne peut être tenu responsable quant à son contenu.


\section{Force majeure}

La responsabilité de \eDevoir ne pourra être engagée en cas de force majeure ou de faits indépendants de sa volonté.


\section{Évolution du présent contrat}

\eDevoir se réserve le droit de modifier les termes, conditions et mentions du présent contrat à tout moment.
Il est ainsi conseillé à l'Utilisateur de consulter régulièrement la dernière version des \textit{Conditions Générales d'Utilisation} disponible sur le site edevoir.com


\section{Durée et résiliation}

Le présent contrat est conclu pour une durée indéterminée à compter de l'Utilisation du Service par l'Utilisateur.


\section{Droit applicable et juridiction compétente}

Les règles en matière de droit, applicables aux contenus et aux transmissions de données sur et autour du site, sont déterminées par la loi française.

L'annulation d'une des dispositions des présentes CGU n'emporte pas la nullité de l'ensemble.

En cas de litige, n'ayant pu faire l'objet d'un accord à l'amiable, seuls les tribunaux français du ressort de la cour d'appel de Lyon sont compétents.
