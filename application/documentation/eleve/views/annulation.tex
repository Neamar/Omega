Tant que vous n’avez pas accepté d’offre, vous pouvez à tout moment annuler votre exercice en cliquant sur le lien “Annuler cet exercice” de la page d’accueil de l’exercice\footnote{Pour trouver cette page, rendez-vous sur \l[/eleve/exercice]{l’accueil des exercices}, puis sélectionnez l’exercice que vous souhaitez annuler dans la \i{liste des exercices en cours}.}. Cette annulation n’engendre évidemment aucun frais.

Une fois l’offre acceptée, vous ne pouvez plus annuler l’exercice. En effet, dès que vous acceptez un exercice, le correcteur peut d’ores et déjà commencer à travailler et, de toutes parts, il est important d’honorer son engagement envers \b{eDevoir}.
Si le litige porte sur une somme importante (demande d’un rapport long par exemple que vous souhaitez annuler après acceptation), vous pouvez \doc[contact]{nous contacter} pour que nous tentions une médiation avec le correcteur.