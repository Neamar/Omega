Une fois connecté, depuis \l[/eleve/]{la page d'accueil élève} cliquez sur "Ajouter un nouvel exercice" (ou suivez directement \l[/eleve/exercice/ajout]{ce lien}).

Sur la page qui s'affiche, vous pouvez rentrer votre nouvel exercice.
\b{Astuce} : si vous avez des exercices indépendants dans une même matière, vous pouvez les poster en plusieurs fois : ainsi, votre travail sera effectué par plusieurs correcteurs en simultané, vous permettant ainsi de gagner du temps ! Ceci ne vous rajoute pas la majoration des __POURCENTAGE_SURACTIVITE__ (voir \doc[eleve/supplement]{Pourquoi un supplément ?}) dans la mesure où vous postez toutes vos demandes avant minuit.

Sur la page, complétez les informations demandées :
\item \b{Le niveau de l'exercice} (le champ est pré-rempli avec le niveau dans lequel vous êtes, vous pouvez modifier cette information sur la \l[/eleve/options/]{page d'options de votre compte} ou le modifier temporairement pour un exercice) ;
\item \b{La matière} de l'exercice (ne postez pas deux exercices de matières distinctes, nous serions obligés de le refuser)
\item \b{La section} : scientifique, littéraire, technologique etc. (le champ est pré-rempli avec la section dans laquelle vous êtes, vous pouvez modifier cette information sur la \l[/eleve/options]{page d'options de votre compte} ou le modifier temporairement pour un exercice) ;
\item \b{Le type de l'exercice}. Il s'agit d'une information à titre indicative pour permettre au correcteur de s'approprier votre exercice d'un coup d'œil. Parmi les possibilités : \i{QCM à justifier}, \i{QCM non justifié}, \i{Exercice court}, \i{Exercice à trou}, \i{Exercice long}, \i{Devoir maison}, \i{Correction de devoir}, \i{Question de cours}, \i{Question théorique}.
\item \b{Le nom de l'exercice}. Donnez ici un nom qui vous permettra de retrouver d'un coup d'œil l'exercice ; par exemple "Correction test physique sur les forces" ou "DM mathématiques de décembre".
\item \b{La demande}. \i{Correction complète} ou \i{Pistes de résolution}. Lorsque vous demandez des pistes de résolution, le correcteur fera en sorte qu'il vous soit possible de répondre à toutes les questions sans pour autant vous livrer la solution sur un plateau. Cette option, très pédagogique, est appréciable si l'on souhaite progresser dans une matière et nous la conseillons vivement. Toutefois, le degré de précision apporté par le correcteur est subjectif, c'est pourquoi nous mettons à votre disposition un espace de discussion privé avec le correcteur afin que vous puissiez, ensemble, éclaircir les points en suspens.
\item \b{Date de rendu} : la date limite à laquelle vous recevrez votre corrigé après avoir accepté une offre d'un correcteur. \doc[eleve/champ_rendu]{Plus d'informations sur la date de rendu}.
\item \b{Date d'annulation automatique} : date limite à laquelle vous souhaitez avoir une proposition de notre site. \doc[eleve/champ_annulation]{Plus d'informations sur la date d'annulation}.
\item \b{Message à l'attention du correcteur} :champ libre pour l'ajout d'informations concernant l'exercice. \doc[eleve/champ_info]{Plus d'informations sur le champ message}.
\item \b{Valeur d'acceptation automatique} : si vous êtes pressés, vous pouvez indiquer une valeur en dessous de laquellle une offre correcteur sera automatiquement acceptée. Voir les \doc[eleve/auto_accept]{conditions spéciales de l'acceptation automatique}. Pour des raisons évidentes, ce montant est strictement confidentiel et ne sera connu d'aucun des correcteurs.

Comme précisé dans les \doc[cgu]{conditions générales d'utilisation}, vous n'êtes pas autorisé à re-poster un exercice tant que celui-ci est en attente de propositions. Tout abus sera sanctionné (\doc[eleve/bloque]{compte bloqué} par exemple).