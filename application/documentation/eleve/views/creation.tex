Une fois connecté, depuis \l[/eleve/]{la page d'accueil élève} cliquez sur "Ajouter un nouvel exercice" (ou suivez directement \l[/eleve/exercice/ajout]{ce lien}).

Sur la page qui s'affiche, vous pouvez rentrer votre nouvel exercice.
\b{Astuce} : si vous avez des exercices indépendants dans une même matière, vous pouvez les poster en plusieurs fois : ainsi, votre travail sera effectué par plusieurs correcteurs en simultané, vous permettant ainsi de gagner du temps ! Ceci ne vous rajoute pas la majoration des %POURCENTAGE_SURACTIVITE%% (voir \doc[eleve/supplement]{Pourquoi un supplément ?}) dans la mesure ou vous postez toutes vos demandes avant minuit.

Sur la page, complétez les informations demandées :
\item \b{Le niveau de l'exercice} (le champ est pré-rempli avec le niveau dans lequel vous êtes, vous pouvez modifier cette information sur la \l[/eleve/options]{page d'options de votre compte} ou le modifier temporairement pour un exercice) ;
\item \b{La matière} de l'exercice (ne postez pas deux exercices de matières distinctes, nous serions obligés de le refuser)
\item \b{La section} : scientifique, littéraire, technologique etc. (le champ est pré-rempli avec la section dans laquelle vous êtes, vous pouvez modifier cette information sur la \l[/eleve/options]{page d'options de votre compte} ou le modifier temporairement pour un exercice) ;
\item \b{Le type de l'exercice}. Il s'agit d'une information à titre indicative pour permettre au correcteur de s'approprier votre exercice d'un coup d'œil. Parmi les possibilités : \i{QCM à justifier}, \i{QCM non justifié}, \i{Exercice court}, \i{Exercice à trou}, \i{Exercice long}, \i{Devoir maison}, \i{Correction de devoir}, \i{Question de cours}, \i{Question théorique}.
\item \b{Le nom de l'exercice}. Donnez ici un nom qui vous permettra de retrouver d'un coup d'œil l'exercice ; par exemple "Correction test physique sur les forces" ou "DM mathématiques de décembre".
\item \b{La demande}. \i{Correction complète} ou \i{Pistes de résolution}. Lorsque vous demandez des pistes de résolution, le correcteur fera en sorte qu'il vous soit possible de répondre à toutes les questions sans pour autant vous livrer la solution sur un plateau. Cette option, très pédagogique, est appréciable si l'on souhaite progresser dans une matière et nous la conseillons vivement. Toutefois, le degré de précision apporté par le correcteur est subjectif, c'est pourquoi nous mettons à votre disposition un espace de discussion privé avec le correcteur afin que vous puissiez, ensemble, éclaircir les points en suspens.
\item \b{Date de rendu} : la date limite à laquelle vous recevrez votre corrigé après avoir accepté une offre d'un correcteur. Si à cette date vous n'aviez rien reçu, vous en seriez alerté par mail et recevriez en remboursement deux fois la somme engagée (bénéfice maximisé par 100€. \doc[eleve/retard]{Consultez l'aide sur les retards pour plus de renseignements}). Pendant tout cet intervalle de temps, il vous est possible d'annuler votre demande de correction d'exercice sans aucun frais (autre que les frais de dépôt dépendant de votre moyen de paiement (cf. \doc[eleve/depot]{déposer de l'argent})).
\item \b{Date d'annulation automatique} : date limite à laquelle vous souhaitez avoir une réponse de notre site. Si vous n'avez accepté aucune offre à cette date -- ou qu'aucune offre ne vous a été faite -- l'exercice sera annulé sans frais. Cette option vous permet de vous prévoir une marge de temps de travail au cas où l'exercice ne trouverait pas preneur. Vous pouvez fixer cette date à même échéance que pour le rendu, mais rien ne vous garantira que d'ici là votre exercice aura été accepté par un correcteur et donc corrigé.
\item \b{Message à l'attention du correcteur} : dans ce champ, libre à vous de rajouter toute information qui vous semblera nécessaire à la correction de votre travail. Vous pourrez par exemple préciser au correcteur que vous n'avez pas le droit d'utiliser telle méthode pour résoudre un exercice, que vous souhaitez que telle réponse soit particulièrement bien détaillée, qu'il n'est pas nécessaire de traiter les dernières questions de l'exercice etc. Cette partie est importante car elle vous permettra de préciser votre souhait et le correcteur la prendra en compte au moment de vous faire une offre (s'il n'a pas à traiter les dernières questions par exemple, il serait logique que le prix baisse en conséquence). Autre cas possible, vous ne souhaitez que la correction de la dernière question de votre DM, précisez-le au correcteur (sa proposition ne s'en trouvera que diminuée). Attention, dans ce cas pensez à préciser au correcteur ce que vous avez trouvé aux questions précédentes, pour lui éviter d'avoir à tout refaire (et donc à vous faire une offre sur-évaluée).
Vous pouvez aussi entrer directement votre exercice dans cette zone s'il est court. Attention : vous pouvez tout écrire dans cette section dans le limite des \doc[cgu]{conditions générales d'utilisation} (aucune information personnelle par exemple ne devra être transmise aux correcteurs sous peine de déclencher une alerte qui pourrait entraîner le \doc[eleve/bloque]{blocage du compte}).
\item \b{Valeur d'acceptation automatique} : si vous êtes pressés, vous pouvez indiquer une valeur en dessous de laquellle une offre correcteur sera automatiquement acceptée. Voir les \doc[eleve/auto_accept]{conditions spéciales de l'acceptation automatique}. Pour des raisons évidentes, ce montant est strictement confidentiel et ne sera connu d'aucun des correcteurs.

Comme précisé dans les \doc[cgu]{conditions générales d'utilisation}, vous n'êtes pas autorisé à re-poster un exercice tant que celui-ci est en attente de propositions. Tout abus sera sanctionné (\doc[eleve/bloque]]{compte bloqué} par exemple).