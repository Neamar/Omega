Lorsqu'une procédure de contestation est lancée, les administrateurs l'examinent et statuent sur son cas. Plusieurs cas de figure peuvent se présenter :
\item \b{Pas de remboursement.} La réclamation nous semble injustifiée et nous décidons qu'il n'y a pas lieu d'effectuer un remboursement. Ce cas se présente typiquement si une question devait ne pas être suffisamment explicitée de la part du correcteur, et que l'élève veuille en profiter pour obtenir remboursement intégral des frais engagés. Vous savez que dans ce cas, une FAQ est tout spécialement ouverte entre vous et le correcteur pour éclaircir de possibles points d'ombre (voir \doc[eleve/faq]{Fonctionnement de la FAQ}). Soyez sûr qu'en aucun cas nous ne prendrons parti pour le correcteur, car c'est la satisfaction du client que nous tenons à tout prix à préserver. Nous ne sommes qu'un intermédiaire entre vous et eux. La seule situation possible pour qu'un remboursement soit refusé serait une demande abusive de la part de l'élève.
\item \b{Remboursement partiel ou total.} Suivant la faute commise par le correcteur, nous pouvons être amené à vous rembourser -- en totalité la plus souvent, ou partiellement en cas de petite erreur. Ce degré d'appréciation est soumis à notre jugement unilatéral (mais vous pouvez \doc[eleve/mecontentement_remboursement]{le contester}).
\item \b{Remboursement supérieur à 100%.} Ce cas peut intervenir si la mauvaise qualité du travail vous a causé un préjudice important (pensez à bien le préciser dans le formulaire de réclamation (voir \doc[eleve/contestation]{procédure de contestation}) ou si la correction présente un contenu inapproprié\footnote{Propos vulgaires ou obscènes. Cette situation, bien que très peu probable, n'est pas à négliger !}.

Soyez assurés que dans un tel cas, nous prendrons toutes les mesures qui s'imposent pour éviter que le correcteur fautif ne récidive. Notre but final reste à tout moment de vous donner pleine satisfaction.