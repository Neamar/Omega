Ce cas est assez peu probable mais, dans un souci de satisfaction, nous l'avons envisagé. Imaginez qu'un correcteur arrive à la fin d'un devoir maison et se rende compte qu'il n'est pas en mesure de répondre à la dernière question : il peut alors décider d'envoyer ce qu'il a réalisé en renonçant à être payé. Dans ce cas, le travail réalisé l'aura été gratuitement et cela, sans pour autant excuser l'incompétence du correcteur, aura le mérite de ne pas vous avoir fait perdre d'argent. Sachez toutefois que même si vous avez bénéficié d'un service gratuit, nous prendrons toutes les mesures côté correcteur afin d'éviter qu'une telle situation ne se reproduise. Encore une fois, cette situation est hautement improbable mais pas impossible.
Le correcteur n'est pas tenu d'envoyer la correction en renonçant à sa paie, donc ne le fera pas forcément. Dans cette hypothèse, si vous deviez recevoir une correction incomplète ou un travail mal réalisé, nous vous incitons fortement à \doc[eleve/contestation]{le contester}.