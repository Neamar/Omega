Fonctionnement d'\eDevoir :
\begin{enumerate}
	\li \b{L'élève s'inscrit.} Il lui suffit pour cela de se rendre sur \l[/eleve/inscription]{la page d'inscription} et d'entrer son mail et son mot de passe : aucune information personnelle n'est demandée.
	\li \b{L'élève valide son compte en cliquant sur le lien reçu dans son mail.}
	\li \b{L'élève se connecte.}
	\li \b{Il poste un devoir.} Il indique la matière, le niveau, le délai, le type de l'exercice (QCM ? Corrigé d'interrogation ratée ? Devoir maison ?) et précise sa demande (Corrigé complet / Pistes de correction). Le support de l'exercice peut être un scan, un texte ou une photo.
	\li \b{Il reçoit une offre.} L'offre est dimensionnée par rapport à son devoir et sa demande. Il peut la refuser, auquel cas une nouvelle offre lui sera transmise (sans garantie de baisse des prix).
	\li \b{L'élève accepte l'offre.} Quand l'offre convient, l'élève l'accepte. Il peut alors vaquer à ses autres occupations intellectuelles, nous travaillons pour lui !
	\li \b{L'élève est informé que son exercice est disponible.} Il peut alors télécharger son exercice et en disposer à sa guise : il lui appartient désormais ! Il a la possibilité de noter le travail du correcteur ou d'effectuer une réclamation dans le cas -- improbable ! -- où l'exercice ne correspond pas à ses besoins.
	\li \b{L'élève peut éclaircir les points restés obscurs en interrogeant le correcteur.} Si l'élève a encore des questions sur l'exercice, il peut les poser à son correcteur jusqu'à une semaine après la date d'expiration de l'exercice.
	\li \b{L'élève confirme que le document est de qualité et note le correcteur.} Une fois la qualité de l'exercice vérifiée, l'élève est en mesure de noter le travail du correcteur. Cette note n'est pas rendue publique sur le site, mais permet en interne de vérifier la fiabilité des personnes que nous engageons, afin d'améliorer notre service et donc la satisfaction des élèves.
\end{enumerate}