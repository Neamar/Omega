Surtout, ne paniquez pas ! La situation est sous contrôle.

Il vous suffit de suivre le lien indiqué dans le mail (ou de cliquer sur l'exercice dont vous voulez consulter la correction dans  \l[/eleve/exercice]{la liste des exercices en cours}).
En cliquant sur "Télécharger le corrigé", vous pourrez récupérer le résultat au format PDF. Utilisez une visionneuse PDF pour pouvoir le lire. Si vous ne pouvez pas ouvrir le fichier, installez \l[http://get.adobe.com/fr/reader/]{Adobe Reader}.

Vous avez lu votre corrigé ?
\item S'il vous a plu, vous pouvez donner une note au correcteur. Cela se passe toujours sur la page d'accueil de l'exercice : il suffit de cliquer sur une étoile pour donner une note. Une seule étoile indique un travail bâclé mais correct, tandis que cinq étoiles indiquent un travail exceptionnel digne de Michel-Ange. Attention, en faisant cela vous "terminez" l'exercice : vous aurez toujours accès au sujet et au corrigé, mais vous ne pourrez plus demander de remboursement\footnote{Vous pourrez tout de même émettre une contestation, mais celle-ci ne donnera pas lieu à un remboursement. Consultez \doc[eleve/contestation_tardive]{cette page} pour plus de détails.}.
\item Si l'exercice contient des fautes ou du contenu inapproprié\footnote{La dernière chanson de Lorie, des insultes...}, est incomplet, mal écrit ou incompréhensible, vous pouvez lancer la procédure de remboursement en cliquant sur le lien "Émettre une réclamation". \doc[eleve/contestation]{Plus d'informations sur les réclamations sur cette page}.