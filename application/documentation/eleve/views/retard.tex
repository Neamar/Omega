Si à l'heure que vous nous aviez indiquée, le document n'est toujours pas disponible dans votre espace élève, vous recevrez automatiquement un mail vous informant de notre incapacité à répondre à votre demande. Vous ne recevrez \b{pas} de corrigé : l'exercice est malheureusement considéré comme un échec. Précisons que cette situation malheureuse n'est absolument pas courante et n'est pas acceptable, mais dans un souci d'exhaustivité nous nous devons de traiter ce cas.
Afin de nous faire pardonner ce manque de professionnalisme, vous recevrez en dédommagement pour le préjudice un remboursement à __POURCENTAGE_RETARD__\% de la somme payée. Afin d'éviter tout abus, la somme gagnée (au delà des 100\% initiaux donc) sera majorée par __MAX_REMBOURSEMENT__€. Le remboursement se fait en points sur votre compte selon l'équivalence standard 1€ = __EQUIVALENCE_POINT__.

Soyez assuré que dans un tel cas, nous prendrons toutes les mesures qui s'imposent pour éviter que le correcteur fautif ne récidive\footnote{Vous n'entendrez plus jamais parler de lui. Enfin si, quand on retrouvera son corps ;)}.

Quelques exemples :
\item Un exercice payé 15€ non livré vous sera remboursé 30€.
\item Un exercice (énorme ! Des pages et des pages !) payé 150€ vous sera remboursé __MAX_REMBOURSEMENT__€.

Vous pourrez retirer les points récupérés par ce remboursement par \doc[eleve/retrait]{la procédure standard de retrait}.