Cette alerte signifie que vous avez communiqué des informations personnelles (identité, numéro de téléphone, adresse mail, etc.) dans une des informations fournies au site (par exemple, et de façon non exhaustive, un élément de la \doc[eleve/FAQ]{FAQ}, le message d'exercice...). Cette attitude est formellement interdite par le site et peut faire l'objet de blocage (voir \doc[eleve/bloque]{compte bloqué}).
\b{eDevoir} fonctionne en double aveugle. Cela signifie que les élèves n'ont pas connaissance de l'identité des correcteurs et vice versa. Ce système vise à rendre le fonctionnement de notre site le plus égalitaire et juste possible.
\item Ceci empêche (entre autre) qu'il y ait des abus au niveau des offres qui sont faites aux élèves et parallèlement, cela prohibe les négociations et la mise en concurrence des correcteurs.
\item Ce système promet également le plus strict anonymat aux élèves, afin qu'ils puissent utiliser \b{eDevoir} en toute tranquillité.