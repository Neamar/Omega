Sur \eDevoir, notre souci principal est de contenter le client -- vous ! C'est pourquoi nous tenons à ce que le travail réalisé par notre équipe de correcteurs soit professionnel et respecte scrupuleusement le niveau de qualité requis. Il n'est donc tout simplement pas acceptable que le travail soit incorrect ou incomplet. Lors de la réception, il vous est possible de nous \doc[eleve/envoye]{envoyer une réclamation}. Nous vous encourageons même à le faire, afin que vous et d'autres ne soyez pas victimes d'une injustice.
Pour cela, cliquez sur le lien présent sur la page d'accueil de l'exercice\footnote{Pour trouver cette page, rendez-vous sur \l[/eleve/exercice]{l'accueil des exercices}, puis cliquez sur l'exercice pour lequel vous souhaitez faire une contestation. Si l'exercice n'apparaît pas dans cette liste, vous l'avez probablement noté comme terminé (ou plus de __DELAI_FAQ__ jours se sont écoulés). Dans ce cas, consultez la page des \doc[eleve/contestation_tardive]{contestation tardives}.} indiquant "Émettre une réclamation".

Dans le formulaire qui s'affiche, remplissez le champ Message en indiquant la raison de votre réclamation. Plus vous serez précis et explicite, plus votre demande de remboursement aura des chances d'aboutir ! N'hésitez pas à fournir des éléments annexes ("en plus d'un travail mal réalisé, le correcteur fait des blagues dans la FAQ et refuse de répondre à mes questions") et à argumenter.
Vous pouvez ajouter des fichiers si vous le désirez (par exemple un scan du corrigé fait en classe) prouvant les erreurs et l'inaptitude du correcteur. La limite du nombre de fichiers est la même que celle lors de l'envoi (__MAX_FICHIERS_EXERCICE__ fichiers).

Cliquez ensuite sur le bouton Envoyer : vous serez informés par mail de la progression de votre demande. \doc[eleve/remboursement]{Consultez les différentes issues possibles}.