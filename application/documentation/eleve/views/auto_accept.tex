Lors de l�\doc[eleve/creation]{ajout d�un exercice}, vous pouvez sp�cifier une valeur en points nomm�e �auto-acceptation�. Cette option permet aux gens press�s de ne pas s�occuper de la gestion des offres :
\item Toute offre qui vous sera faite de co�t inf�rieur ou �gal � cette valeur sera automatiquement accept�e.
\item Toute offre qui vous sera faite de co�t sup�rieur � cette valeur sera automatiquement refus�e (dans la limite des %MAX_REFUS% refus, voir les \doc[eleve/refus]{informations sur les refus}).

La valeur de l�acceptation automatique ne peut pas d�passer la somme actuelle que vous poss�dez sur votre compte eDevoir.
� tout moment, la valeur de l�auto acceptation est maximis�e par votre solde eDevoir.
Autrement dit, si vous faites descendre votre compte d�une fa�on quelconque (\doc[eleve/retrait]{retrait}, \doc[eleve/acceptation]{acceptation d�un autre exercice}...) et que votre nouveau solde ne permet plus d�utiliser l�auto-acceptation initiale, les offres sup�rieures � votre solde seront refus�es.
Si vous rajoutez des points sur votre solde et red�passez l�auto-accept, c�est cette valeur d�auto-accept qui est � nouveau prise en compte.

\subsection{Limitation des abus}
Bien entendu, les correcteurs ne savent pas quelle valeur vous indiquez comme auto-accept, ce qui �vite les abus du type �proposer la somme max�. Il s�agit d�un jeu de propositions/acceptations en double aveugle, �quitable pour tout le monde.