Lors de l'\doc[eleve/creation]{ajout d'un exercice}, vous pouvez spécifier une valeur en points nommée "auto-acceptation". Cette option permet aux gens pressés de ne pas s'occuper de la gestion des offres :
\item Toute offre qui vous sera faite de coût inférieur ou égal à cette valeur sera automatiquement acceptée.
\item Toute offre qui vous sera faite de coût supérieur à cette valeur sera automatiquement refusée (dans la limite des __MAX_REFUS__ refus, voir les \doc[eleve/refus]{informations sur les refus}).

La valeur de l'acceptation automatique ne peut pas dépasser la somme actuelle que vous possédez sur votre compte \eDevoir.
À tout moment, la valeur de l'auto acceptation est maximisée par votre solde \eDevoir.
Autrement dit, si vous faites descendre votre compte d'une façon quelconque (\doc[eleve/retrait]{retrait}, \doc[eleve/acceptation]{acceptation d'un autre exercice}...) et que votre nouveau solde ne permet plus d'utiliser l'auto-acceptation initiale, les offres supérieures à votre solde seront refusées.
Si vous rajoutez des points sur votre solde et redépassez l'auto-accept, c'est cette valeur d'auto-accept qui est à nouveau prise en compte.

\subsection{Limitation des abus}
Bien entendu, les correcteurs ne savent pas quelle valeur vous indiquez comme auto-accept, ce qui évite les abus du type "proposer la somme max". Il s'agit d'un jeu de propositions/acceptations en double aveugle, équitable pour tout le monde.