Lors de la création d'un exercice, il est demandé de saisir la \b{date de rendu}.

Il s'agit de la date limite à laquelle vous recevrez votre corrigé après avoir accepté une offre d'un correcteur.
Si à cette date vous n'aviez rien reçu (mais que vous avez payé un correcteur), vous en seriez alerté par mail et recevriez en remboursement deux fois la somme engagée (voir conditions sur \doc[eleve/retard]{les retards}).

Pendant tout cet intervalle de temps, et tant que vous n'avez accepté aucune offre, il vous est possible d'annuler votre demande de correction d'exercice sans aucun frais (autre que les frais de dépôt dépendant de votre moyen de paiement (cf. \doc[eleve/depot]{déposer de l'argent}).

En résumé, si la date est dépassée, 4 possibilités :
\begin{enumerate}
\li Vous avez accepté une offre d'un correcteur avant, et vous avez reçu le corrigé dans les temps. Tout va bien ; la procédure standard s'applique (et vous disposez de __DELAI_FAQ__  jours pour \doc[eleve/contestation]{émettre une réclamation} si nécessaire).
\li Vous avez accepté une offre et vous n'avez toujours rien reçu. Vous êtes alors remboursés au double  (voir conditions sur \doc[eleve/retard]{les retards}).
\li Vous n'avez jamais accepté aucune offre ; l'exercice est terminé.
\li Vous n'avez jamais reçu aucune offre ; l'exercice est terminé.
\end{enumerate}

Une autre date vous est demandée, la \doc[eleve/champ_annulation]{date d'annulation automatique}.