Le \b{message à l'attention du correcteur} est un champ permettant de transmettre des informations utiles à la résolution de l'exercice.

Libre à vous de rajouter toute information qui vous semblera nécessaire à la correction de votre travail. Vous pourrez par exemple préciser au correcteur que vous n'avez pas le droit d'utiliser telle méthode pour résoudre un exercice, que vous souhaitez que telle réponse soit particulièrement bien détaillée, qu'il n'est pas nécessaire de traiter les dernières questions de l'exercice etc. Cette partie est importante car elle vous permettra de préciser votre souhait et le correcteur la prendra en compte au moment de vous faire une offre (s'il n'a pas à traiter les dernières questions par exemple, il serait logique que le prix baisse en conséquence). Autre cas possible, vous ne souhaitez que la correction de la dernière question de votre DM, précisez-le au correcteur (sa proposition ne s'en trouvera que diminuée). Attention, dans ce cas pensez à préciser au correcteur ce que vous avez trouvé aux questions précédentes, pour lui éviter d'avoir à tout refaire (et donc à vous faire une offre sur-évaluée).

Vous pouvez aussi entrer directement votre exercice dans cette zone s'il est court.

Attention : vous pouvez tout écrire dans cette section dans le limite des \doc[cgu]{conditions générales d'utilisation} (aucune information personnelle par exemple ne devra être transmise aux correcteurs sous peine de déclencher une alerte qui pourrait entraîner le \doc[eleve/bloque]{blocage du compte}).