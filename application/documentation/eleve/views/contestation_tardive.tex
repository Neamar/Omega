Si vous avez mis une note au correcteur, ou si vous avez laissé passer plus de %DELAI_FAQ% jours après la date d'expiration de l'exercice, celui-ci est clos : vous pouvez encore consulter sujet et corrigé, mais vous ne pouvez plus faire d'actions dessus.

Le formulaire de remboursement est donc théoriquement clos. Cependant, il peut arriver que vous receviez une correction en retard ou que vous ayez fait une fausse manœuvre. Nous laissons alors à votre disposition le formulaire de contestation, sans toutefois qu'un remboursement en points soit possible.
Malgré cela, nous vous encourageons à effectuer cette démarche si vous découvriez des erreurs : cela nous permettra de vérifier la cohérence du travail d'un correcteur et d'effectuer les actions nécessaires.

Pour trouver ce formulaire, rendez-vous sur la page \l[/eleve/exercices]{d'accueil des exercices} puis recherchez la liste "Exercices terminés".
Cliquez sur l'exercice qui vous intéresse, puis sur le lien "Émettre une réclamation".
La marche à suivre est alors similaire à celle décrite sur la page \doc[eleve/contestation]{de contestation}.