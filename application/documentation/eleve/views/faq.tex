Dès que vous \doc[eleve/acceptation]{acceptez une offre}, la FAQ de l'exercice s'ouvre. Elle se fermera %DELAI_FAQ% jours après la date d'expiration spécifiée.

Cette FAQ permet à l'élève et au correcteur de poser des questions et d'obtenir des réponses (\doc[Tex]{comment rédiger des formules mathématiques dans la FAQ ?}).
Elle restera ouverte jusqu'à %DELAI_FAQ% jours après votre date d'expiration de l'exercice : au delà, elle sera fermée et vous ne pourrez plus poser de nouvelles questions.
Le correcteur pourra l'utiliser pour vous poser des questions sur l'exercice. Vous recevrez alors un mail vous incitant à répondre à la question afin de débloquer le correcteur.
Vous pouvez vous aussi poser vos questions à tout moment, et votre correcteur fera de son mieux pour y répondre.
Attention : cette FAQ n'est pas un soutien scolaire ! Vos questions doivent être en rapport avec l'exercice demandé.

Restez courtois et agréable, cette FAQ est là pour vous aider !
Si un correcteur ne répond pas à vos questions (après un délai raisonnable) ou y répond mal de façon répétitive, vous pouvez cliquez sur le bouton "Signaler ce message" pour nous faire parvenir vos doléances.

Le langage SMS y est prohibé : nous ne lisons que le français correct.