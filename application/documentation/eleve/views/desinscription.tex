Vous pouvez à tout moment vous désinscrire de \eDevoir\footnote{Nous ne retenons personne captif !}. Pour cela, il vous suffit de vous rendre à la page \l[/eleve/desinscription]{désinscription} de votre espace personnel et de cliquer sur le bouton "Me désinscrire". Dès lors, vous serez considéré comme ne faisant plus partie des membres de \eDevoir. Attention, vous devez avoir retiré tous vos points restant avant de vous désinscrire, sans quoi vous perdrez ce reliquat\footnote{Un message vous préviendra au moment de la désinscription s'il reste de l'argent sur votre compte \eDevoir}. Pour cela, il vous suffit au préalable de vous connecter à la page des \l[/eleve/retrait]{retraits} (\doc[eleve/retrait]{aide sur les retraits}).

\subsection{Et concrètement, qu'implique la désinscription ?}
Une fois désinscrit, vous ne pourrez plus rouvrir de compte sous cette adresse mail.
De plus, la connexion sera impossible -- le compte est complètement bloqué.
Si vous souhaitez un jour réutiliser le site, vous devrez nous contacter -- nous vérifierons alors que tout est en ordre avant de rétablir les accès.

Notez que la désinscription n'entraîne pas la suppression des exercices que vous avez crées, ceux-ci restent présents sur le serveur, mais les liens pour les retrouver ne sont plus disponibles. Pensez donc à tout sauvegarder ou à enregistrer les url avant de vous désinscrire !