Lorsque vous recevez une offre, vous pouvez la consulter directement depuis le lien dans le mail vous signalant l'offre. Vous pouvez aussi y accéder depuis la page d'accueil de l'exercice\footnote{Pour trouver cette page, rendez-vous sur \l[/eleve/exercice]{l'accueil des exercices}, puis sélectionnez l'exercice dont vous souhaitez consulter l'offre dans la \i{liste des exercices en cours}.}. en cliquant sur le lien "Voir l'offre du correcteur".

Lors de la consultation d'une offre, vous pouvez :
\item \b{Accepter l'offre}. Cette option n'est disponible que si votre compte possède assez de points pour accepter l'offre.
\item \b{Refuser l'offre}. L'exercice repart alors dans la "foire aux exercices" des correcteurs. Le correcteur qui vient de voir son offre refusée ne peut plus enchérir sur cet exercice. Attention : vous n'avez aucune garantie que l'offre suivante sera plus avantageuse. En effet, les correcteurs ne se connaissent pas et font leur offre chacun de leur côté, sans savoir ce que les autres proposent.
\item \b{Annuler l'exercice}. Vous pouvez tout simplement annuler l'exercice. Ceci sera bien-entendu fait sans aucun frais.

Si vous refusez %MAX_REFUS% fois une offre, l'exercice est automatiquement annulé (sans frais). Vous pouvez le réenvoyer si vous le souhaitez.